\documentclass[../main.tex]{subfiles}
\graphicspath{{\subfix{../images/}}}
\begin{document}

\chapter{Background}

Recent advances in legged locomotion have largely focused on either implicit gait-based policies or perception-driven foothold selection modules. However, these approaches often trade off between agility, expressiveness, and computational cost. In contrast to gaited methods that impose rhythmic structure, non-gaited planning allows for more versatile, terrain-adaptive behaviors—but typically at the expense of increased planning complexity. Bridging this gap, our proposed approach draws on greedy methods for their computational efficiency and on convolutional neural networks (CNNs) for terrain-aware generalization, aiming to achieve dynamic, non-gaited footstep planning in real time. This section reviews the foundations upon which our approach builds: learning-based locomotion strategies, footstep planners, greedy selection mechanisms, and foothold classification techniques. Each informs a different aspect of our planner's design and situates our method within the broader landscape of quadruped control.

\section{Locomotion Planning Paradigms}

\begin{outline}
  Categorize and discuss existing approaches to legged locomotion, including gait-based and non-gaited methods.
\end{outline}

\section{Learning-Based Locomotion Strategies}
\label{sec:learning-based-locomotion-strategies}

\begin{outline}
  Review recent advances in deep learning for locomotion control,
  highlighting the works of Shi et al., Xie et al., and others.
  Discuss how they often rely on fixed or implicit gaits.
\end{outline}

Deep learning has shown strong potential for generating agile, robust
locomotion policies, often without explicit footstep planning. Shi et
al. \cite{shi_terrain-aware_2023} modulate trajectory generator
parameters in real time for energy-efficient walking, while Xie et
al. \cite{xie_glide_2023} train RL policies on centroidal dynamics
models to output body accelerations under fixed gait heuristics. Duan
et al. \cite{duan_sim--real_2022} learn step-to-step transitions
using proprioception, and Siekmann et al. \cite{siekmann_blind_2021}
achieve blind stair climbing through LSTM-based proprioceptive
policies. Lee et al. \cite{lee_learning_2020} infer terrain structure
from proprioceptive history using a temporal CNN and automated
curriculum learning. Though effective, these methods rely on fixed or
implicit gait patterns and lack explicit control over individual footsteps.

Subsequent works extend these ideas through high-level gait
selection. Da et al. \cite{da_learning_2020} use a DQN to choose
among predefined gait primitives executed by a low-level controller,
while Yang et al. \cite{yang_fast_2021} learn policies that output
gait parameters—frequency, swing ratio, and phase offsets—interpreted
by a phase integrator. Both enable efficient gait modulation but
assume flat terrain and heuristic foot placement. Sun et al.
\cite{sun_online_2024} integrate offline gait optimization with
contact-implicit trajectory optimization and high-frequency MPC,
achieving dynamic control but limiting adaptability to discrete,
speed-dependent gaits.

In contrast, Zhang et al. \cite{zhang_learning_2023} propose an
end-to-end RL approach mapping proprioceptive input directly to joint
targets, reaching 2.5 m/s and generalizing across terrains via a
terrain curriculum and curiosity rewards. However, such models
require slow training, and re-training for new environments. Zhang et
al. \cite{zhang_novel_2024} address multi-gait transitions using
heuristically defined tables between similar gaits, improving
flexibility but still constraining behaviors to predefined families.

Overall, learning-based locomotion methods trade interpretability and
efficiency for expressiveness and adaptability. While
gait-conditioned policies achieve reliable control, they lack the
flexibility needed for explicit, non-gaited footstep planning.

\section{Learning-Based Footstep Planners}

\begin{outline}
  Detail the evolution of learned footstep planners, including FootstepNet, ContactNet, and DeepLoco. Critically analyze their limitations, particularly regarding simultaneous support for non-gaited and dynamically unstable footstep generation.
\end{outline}

Learning-based footstep planners combine perception and control through data-driven models, often using a learned component upstream of a model-predictive controller. FootstepNet \cite{gaspard_footstepnet_2024} uses deep reinforcement learning to generate heuristic-free plans but is limited to bipeds in simple 2D environments. ContactNet \cite{bratta_contactnet_2024} employs a CNN to produce non-gaited footstep sequences using greedy selection, though it is constrained to one-leg-at-a-time motion and coarse foothold discretization. DeepGait \cite{tsounis_deepgait_2020} separates footstep planning and motion optimization, enabling more modular design but still relying on static gaits. Omar et al. \cite{omar_fast_2022} use a CNN to identify safe stepping regions, which are then decomposed into convex regions and passed to an MPC for downstream planning. While efficient and fully onboard, the method relies on fixed gait sequences and is primarily focused on safe terrain classification. Villarreal et al. \cite{villarreal_fast_2019} similarly use a CNN for foothold classification, replacing expert heuristics and enabling real-time performance with significantly faster inference, but their method also depends on a fixed gait schedule. DeepLoco \cite{peng_deeploco_2017} introduces a hierarchical policy for footstep and motion planning, but its coarse terrain input and focus on gaited footstep generation for bipeds limit dynamic adaptability. Taouil et al. \cite{taouil_non-gaited_2025} use Monte Carlo Tree Search (MCTS) to explore dynamic and non-gaited stepping combinations, making them first to generate dynamic, non-gaited footstep plans in real time. Their approach does suffer from a high computational cost due to its high branching factor, especially on certain types of terrain \cite{bratta_contactnet_2024}. Overall, while these approaches advance learning-based planning, few simultaneously support both non-gaited and dynamically unstable footstep generation.

\section{Greedy and Heuristic Footstep Planners}

\begin{outline}
  Explore the use of greedy and heuristic methods in locomotion,
  referencing works like those by Gao et al. and Zucker et al.
  Explain their benefits (computational efficiency) and how your work
  extends them.
\end{outline}

Greedy planners offer computationally efficient alternatives to
exhaustive search, often prioritizing goal-directed heuristics or
local stability metrics. Gao et al. \cite{gao_global_2024} propose
GH-QP, a greedy-heuristic hybrid that incorporates expected robot
speed into a footstep planning heuristic, though it is limited to
bipedal systems and lacks support for dynamic footstep plans. Zucker
et al. \cite{zucker_optimization_2011} use A* with a learned terrain
cost and Dubins car heuristic for footstep planning, enabling
effective search in SE(2) but restricted to static, gated locomotion.
Kalakrishnan et al. \cite{kalakrishnan_learning_2011} combine greedy
terrain-cost search with a coarse-to-fine pose optimization pipeline,
but rely on fixed gait patterns and struggle with highly constrained
environments. While these works demonstrate the potential of greedy
methods, none support both dynamically unstable and non-gaited
footstep generation. Notably, some prior methods already
discussed—such as ContactNet \cite{bratta_contactnet_2024}, DeepLoco
\cite{peng_deeploco_2017}, and Villarreal et al.
\cite{villarreal_fast_2019}—also incorporate greedy elements in their pipelines.

\section{Foothold Classification Algorithms}

\begin{outline}
  Summarize the role of perception-based methods for identifying safe
  stepping regions, such as those by Asselmeier et al. and Omar et
  al., and explain how these are often separate from the full motion
  planning process.
\end{outline}

Foothold classification algorithms focus on determining safe stepping
regions, often using learned terrain models to support downstream
planners that enforce safety constraints. Asselmeier et al.
\cite{asselmeier_steppability-informed_2024} generate steppability
maps from visual inputs to guide a trajectory-optimization-based
planner, validating their perception-to-control pipeline in
simulation. Omar et al. \cite{omar_safesteps_2023} propose a
two-stage classifier that first identifies steppable regions before
selecting footholds, emphasizing safety over expressiveness by
evaluating only candidate swing-foot locations. Similarly, Omar et
al. \cite{omar_fast_2022} use a CNN to identify safe footholds, which
are grouped into convex clusters and passed to an MPC for real-time
planning, but the approach relies on fixed gait sequences. These
methods provide robust terrain understanding but are typically
designed as standalone perception modules, separate from the full
footstep generation process, and thus are not directly applicable to
dynamic or non-gaited motion planning.

\section{Synthesis}

\begin{outline}
  Conclude the chapter by synthesizing the prior work and clearly
  positioning your proposed method as a bridge between these
  different approaches, addressing the specific gaps you identified.
\end{outline}


\end{document}
