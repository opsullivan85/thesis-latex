\section{End-to-End Neural Networks}
\label{sec:background-end-to-end-neural-net}

End-to-end neural network approaches for legged locomotion directly
map sensory inputs to control outputs, bypassing explicit modeling of
dynamics or footstep planning.

Zhang et al. \cite{zhang_learning_2023} propose an end-to-end RL
approach mapping proprioceptive input directly to joint targets,
reaching 2.5 m/s and generalizing across terrains via a terrain
curriculum and curiosity rewards. However, such models require slow
training, and re-training for new environments. Zhang et al.
\cite{zhang_novel_2024} address multi-gait transitions using
heuristically defined tables between similar gaits, improving
flexibility but still constraining behaviors to predefined families.

\begin{todo}
  add another paragraph of references here
\end{todo}

While these models achieve impressive feats, they often require
extensive training and struggle to generalize to unseen terrains or
tasks. This motivates approaches that combine the adaptability of
learning-based methods with the structure and guarantees of
traditional control, enabling fast, terrain-adaptive footstep
planning without the need for extensive retraining.
