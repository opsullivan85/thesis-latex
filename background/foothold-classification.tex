\section{Foothold Classification Algorithms}

\begin{outline}
  Summarize the role of perception-based methods for identifying safe
  stepping regions, such as those by Asselmeier et al. and Omar et
  al., and explain how these are often separate from the full motion
  planning process.
\end{outline}

Foothold classification algorithms focus on determining safe stepping
regions, often using learned terrain models to support downstream
planners that enforce safety constraints. Asselmeier et al.
\cite{asselmeier_steppability-informed_2024} generate steppability
maps from visual inputs to guide a trajectory-optimization-based
planner, validating their perception-to-control pipeline in
simulation. Omar et al. \cite{omar_safesteps_2023} propose a
two-stage classifier that first identifies steppable regions before
selecting footholds, emphasizing safety over expressiveness by
evaluating only candidate swing-foot locations. Similarly, Omar et
al. \cite{omar_fast_2022} use a CNN to identify safe footholds, which
are grouped into convex clusters and passed to an MPC for real-time
planning, but the approach relies on fixed gait sequences. These
methods provide robust terrain understanding but are typically
designed as standalone perception modules, separate from the full
footstep generation process, and thus are not directly applicable to
dynamic or non-gaited motion planning.
