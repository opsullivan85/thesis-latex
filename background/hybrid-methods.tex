\section{Hybrid Methods}
\label{sec:background-hybrid-methods}

Hybrid methods for legged locomotion combine elements of traditional
control (e.g., optimization or Model Predictive Control, MPC) and
learning-based approaches to achieve fast, terrain-adaptive footstep
generation. These methods often solve high-level planning problems
with learned models, while delegating low-level control to
traditional, high-frequency solvers. This division of labor enables
real-time performance without sacrificing the expressiveness offered
by deep learning.

\subsection{Learned Policies for Gait and Trajectory Modulation}
\label{sec:hybrid-gait-policies}

A major class of hybrid methods utilizes deep learning to generate
agile, robust locomotion policies, often focusing on body motion and
gait modulation rather than explicit, step-by-step foot placement.
These approaches typically learn a mapping from sensor data to
control commands, relying on fixed or implicit gait patterns.

Deep learning has shown strong potential for generating such robust
policies. Shi et al. \cite{shi_terrain-aware_2023} modulate
trajectory generator parameters in real time for energy-efficient
walking, while Xie et al. \cite{xie_glide_2023} train Reinforcement
Learning (RL) policies on centroidal dynamics models to output body
accelerations under fixed gait heuristics. Other methods focus on
blind, proprioceptive-based control: Duan et al.
\cite{duan_sim--real_2022} learn step-to-step transitions using
proprioception, and Siekmann et al. \cite{siekmann_blind_2021}
achieve blind stair climbing through LSTM-based proprioceptive
policies. Lee et al. \cite{lee_learning_2020} even infer terrain
structure from proprioceptive history using a temporal CNN and
automated curriculum learning.

Subsequent works extend these ideas through explicit high-level gait
selection. Da et al. \cite{da_learning_2020} use a Deep Q-Network
(DQN) to choose among predefined gait primitives executed by a
low-level controller, while Yang et al.
\cite{yang_fast_2021} learn policies that output gait
parameters---frequency, swing ratio, and phase offsets---interpreted
by a phase integrator. Both enable efficient gait modulation but
generally assume flat terrain and heuristic foot placement. Sun et
al. \cite{sun_online_2024} integrate offline gait optimization with
contact-implicit trajectory optimization and high-frequency MPC,
achieving dynamic control but limiting adaptability to a set of
discrete, speed-dependent gaits.

Although effective at generating robust body motions, these methods
primarily rely on fixed or implicit gait patterns and largely lack
explicit, flexible control over individual footsteps based on local
terrain features.

\subsection{Learning-Based Footstep Planners}
\label{sec:hybrid-footstep-planners}

In contrast to gait modulation, learning-based footstep planners
focus on explicitly learning the footstep planning component. These
methods combine perception and control through data-driven models,
often coupling the learned components with downstream MPC or
search-based planning to enforce dynamic feasibility.

Early methods demonstrated the concept but faced scope limitations.
FootstepNet \cite{gaspard_footstepnet_2024} generates heuristic-free
plans via deep RL but is limited to bipeds in simple 2D environments.
DeepLoco \cite{peng_deeploco_2017} introduces a hierarchical policy
for footstep and motion planning, yet its coarse terrain input and
biped focus limit dynamic adaptability. Other approaches separate the
planning concerns: DeepGait \cite{tsounis_deepgait_2020} separates
footstep planning from motion optimization, enabling modularity but
relying on static gaits. RLOC \cite{gangapurwala_rloc_2022} uses
a convolutional encoder to pass terrain data embeddings to multiple
RL modules before generating joint torques with a traditional whole
body controller. One of these RL modules is a footstep planner,
but the system uses a fixed gait schedule and the footstep planner
is only evaluated once at the beginning of each cycle.

A key challenge is moving beyond fixed gaits while maintaining
stability. ContactNet \cite{bratta_contactnet_2024} uses a CNN to
produce non-gaited footstep sequences with greedy selection, though
it is constrained to one-leg-at-a-time motion and coarse foothold
discretization. Omar et al. \cite{omar_fast_2022} and Villarreal et
al. \cite{villarreal_fast_2019} use CNNs for terrain classification
to inform MPC or heuristic-free selection, but both depend on fixed
gait sequences.

\subsubsection{Recent Advances in Terrain Generalization}

More recent advances expand terrain generalization and multi-contact
reasoning. Tolomei et al. \cite{tolomei_learning_2025} created
a framework generalized to 3D terrain and variable foot shapes,
ranking footholds based on terrain curvature, but still follow a
fixed gait. Chen et al. \cite{chen_gait_2024} apply a DQN to a
hexapod, simultaneously selecting both next foot positions and gait,
with fallback to predefined behaviors outside expected states, but
motion is constrained to three-leg-at-a-time gaits. Similarly, Yao et
al. \cite{yao_hierarchical_2021} output both swing leg motions and
desired robot pose from a deep RL policy but remain limited to a
trotting gait. Meduri et al. \cite{meduri_deepq_2021} use a DQN to
learn 3D foothold cost maps, but retain fixed biped gait timing.

Overall, these approaches highlight trade-offs between expressive,
non-gaited footstep planning and computational efficiency. CNN-based
and RL methods improve terrain generalization, but fixed gaits or
sequential leg motions remain common constraints for achieving
real-time performance and maintaining stability.

\subsection{Greedy Planning and Learned Foothold Classification}
\label{sec:hybrid-greedy-classification}

To complement learning-based approaches, other hybrid methods focus
on improving the efficiency of the planning step itself through
greedy search or by leveraging learned terrain models for safe
foothold identification.

\subsubsection{Greedy Planners for Efficiency}

Greedy planners offer computationally efficient alternatives to
exhaustive search, often prioritizing goal-directed heuristics or
local stability metrics. Gao et al. \cite{gao_global_2024} propose
GH-QP, a greedy-heuristic hybrid that incorporates expected robot
speed into a footstep planning heuristic, though it is limited to
bipedal systems and lacks support for dynamic footstep plans. Zucker
et al. \cite{zucker_optimization_2011} use A* with a learned terrain
cost and Dubins car heuristic for footstep planning, enabling
effective search in SE(2) but restricted to static, gated locomotion.
Kalakrishnan et al. \cite{kalakrishnan_learning_2011} combine greedy
terrain-cost search with a coarse-to-fine pose optimization pipeline,
but rely on fixed gait patterns.

It is worth noting that greedy elements are already incorporated into
several learning-based methods discussed previously, such as
ContactNet \cite{bratta_contactnet_2024}, DeepLoco
\cite{peng_deeploco_2017}, and Villarreal et al.
\cite{villarreal_fast_2019}. While these works demonstrate the
potential of greedy methods for efficiency, none explicitly support
both dynamically unstable and non-gaited footstep generation.

\subsubsection{Learned Foothold Classification}

A related line of research focuses solely on determining safe
stepping regions (foothold classification) using learned terrain
models to support downstream planners that enforce safety
constraints. Asselmeier et al.
\cite{asselmeier_steppability-informed_2024} generate steppability
maps from visual inputs to guide a trajectory-optimization-based
planner. Omar et al. \cite{omar_safesteps_2023} propose a two-stage
classifier that first identifies steppable regions before selecting
footholds, emphasizing safety. Similarly, Omar et al.
\cite{omar_fast_2022} use a CNN to identify safe footholds, which are
then passed to an MPC for real-time planning, but the approach relies
on fixed gait sequences.

These methods provide robust terrain understanding but are typically
designed as standalone perception modules, separate from the full
dynamic footstep generation process, and thus are not directly
applicable to dynamic or non-gaited motion planning without
significant integration.
