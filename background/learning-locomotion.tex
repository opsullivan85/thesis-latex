\section{Learning-Based Locomotion Strategies}
\label{sec:background-learning-based-locomotion-strategies}

\begin{outline}
  Review recent advances in deep learning for locomotion control,
  highlighting the works of Shi et al., Xie et al., and others.
  Discuss how they often rely on fixed or implicit gaits.
\end{outline}

Deep learning has shown strong potential for generating agile, robust
locomotion policies, often without explicit footstep planning. Shi et
al. \cite{shi_terrain-aware_2023} modulate trajectory generator
parameters in real time for energy-efficient walking, while Xie et
al. \cite{xie_glide_2023} train RL policies on centroidal dynamics
models to output body accelerations under fixed gait heuristics. Duan
et al. \cite{duan_sim--real_2022} learn step-to-step transitions
using proprioception, and Siekmann et al. \cite{siekmann_blind_2021}
achieve blind stair climbing through LSTM-based proprioceptive
policies. Lee et al. \cite{lee_learning_2020} infer terrain structure
from proprioceptive history using a temporal CNN and automated
curriculum learning. Though effective, these methods rely on fixed or
implicit gait patterns and lack explicit control over individual footsteps.

Subsequent works extend these ideas through high-level gait
selection. Da et al. \cite{da_learning_2020} use a DQN to choose
among predefined gait primitives executed by a low-level controller,
while Yang et al. \cite{yang_fast_2021} learn policies that output
gait parameters—frequency, swing ratio, and phase offsets—interpreted
by a phase integrator. Both enable efficient gait modulation but
assume flat terrain and heuristic foot placement. Sun et al.
\cite{sun_online_2024} integrate offline gait optimization with
contact-implicit trajectory optimization and high-frequency MPC,
achieving dynamic control but limiting adaptability to discrete,
speed-dependent gaits.

In contrast, Zhang et al. \cite{zhang_learning_2023} propose an
end-to-end RL approach mapping proprioceptive input directly to joint
targets, reaching 2.5 m/s and generalizing across terrains via a
terrain curriculum and curiosity rewards. However, such models
require slow training, and re-training for new environments. Zhang et
al. \cite{zhang_novel_2024} address multi-gait transitions using
heuristically defined tables between similar gaits, improving
flexibility but still constraining behaviors to predefined families.

Overall, learning-based locomotion methods trade interpretability and
efficiency for expressiveness and adaptability. While
gait-conditioned policies achieve reliable control, they lack the
flexibility needed for explicit, non-gaited footstep planning.
