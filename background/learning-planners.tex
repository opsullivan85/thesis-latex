\section{Learning-Based Footstep Planners}

\begin{outline}
  Detail the evolution of learned footstep planners, including
  FootstepNet, ContactNet, and DeepLoco. Critically analyze their
  limitations, particularly regarding simultaneous support for
  non-gaited and dynamically unstable footstep generation.
\end{outline}

Learning-based footstep planners combine perception and control
through data-driven models, often using a learned component upstream
of a model-predictive controller. FootstepNet
\cite{gaspard_footstepnet_2024} uses deep reinforcement learning to
generate heuristic-free plans but is limited to bipeds in simple 2D
environments. ContactNet \cite{bratta_contactnet_2024} employs a CNN
to produce non-gaited footstep sequences using greedy selection,
though it is constrained to one-leg-at-a-time motion and coarse
foothold discretization. DeepGait \cite{tsounis_deepgait_2020}
separates footstep planning and motion optimization, enabling more
modular design but still relying on static gaits. Omar et al.
\cite{omar_fast_2022} use a CNN to identify safe stepping regions,
which are then decomposed into convex regions and passed to an MPC
for downstream planning. While efficient and fully onboard, the
method relies on fixed gait sequences and is primarily focused on
safe terrain classification. Villarreal et al.
\cite{villarreal_fast_2019} similarly use a CNN for foothold
classification, replacing expert heuristics and enabling real-time
performance with significantly faster inference, but their method
also depends on a fixed gait schedule. DeepLoco
\cite{peng_deeploco_2017} introduces a hierarchical policy for
footstep and motion planning, but its coarse terrain input and focus
on gaited footstep generation for bipeds limit dynamic adaptability.
Taouil et al. \cite{taouil_non-gaited_2025} use Monte Carlo Tree
Search (MCTS) to explore dynamic and non-gaited stepping
combinations, making them first to generate dynamic, non-gaited
footstep plans in real time. Their approach does suffer from a high
computational cost due to its high branching factor, especially on
certain types of terrain \cite{bratta_contactnet_2024}. Overall,
while these approaches advance learning-based planning, few
simultaneously support both non-gaited and dynamically unstable
footstep generation.
