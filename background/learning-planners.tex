\section{Learning-Based Footstep Planners}
\label{sec:learning-based-footstep-planners}

Learning-based footstep planners combine perception and control
through data-driven models, often coupling learned components with
downstream MPC or search-based planning. FootstepNet
\cite{gaspard_footstepnet_2024} generates heuristic-free plans via
deep RL but is limited to bipeds in simple 2D environments.
ContactNet \cite{bratta_contactnet_2024} uses a CNN to produce
non-gaited footstep sequences with greedy selection, though it is
constrained to one-leg-at-a-time motion and coarse foothold
discretization. DeepGait \cite{tsounis_deepgait_2020} separates
footstep planning from motion optimization, enabling modularity but
relying on static gaits. Omar et al. \cite{omar_fast_2022} and
Villarreal et al. \cite{villarreal_fast_2019} use CNNs for terrain
classification to inform MPC or heuristic-free selection, but both
depend on fixed gait sequences. DeepLoco \cite{peng_deeploco_2017}
introduces a hierarchical policy for footstep and motion planning,
yet its coarse terrain input and biped focus limit dynamic adaptability.

Recent advances further expand terrain generalization and
multi-contact reasoning. Tolomei et al.
\cite{tolomei_learning-based_2025} create a framework like ContactNet
but generalized to 3D terrain and variable foot shapes, ranking
footholds based on terrain curvature, foot parameters, and kinematic
feasibility, but still follow a fixed gait. Chen et al.
\cite{chen_gait_2024} apply a DQN to a hexapod, simultaneously
selecting next foot positions and gait, with fallback to predefined
behaviors outside expected states, but motion is constrained to
three-leg-at-a-time gaits. Yao et al. \cite{yao_hierarchical_2021}
output both swing leg motions and desired robot pose from a deep RL
policy, demonstrating accurate control with a small network, yet
remain limited to a trotting gait. Meduri et al.
\cite{meduri_deepq_2021} use a Deep Q Network to learn 3D foothold
cost maps, similar to ContactNet, but retain fixed biped gait timing.

Overall, these approaches highlight trade-offs between expressive,
non-gaited footstep planning and computational efficiency. CNN-based
and RL methods improve terrain generalization and allow multi-step
planning, but fixed gaits or sequential leg motions remain common
constraints. This motivates hybrid methods that combine learned
foothold evaluation with fast, greedy planning to achieve real-time,
non-gaited footstep generation—forming the basis of our proposed approach.
