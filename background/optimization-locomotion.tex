\section{Optimization-Based Locomotion Strategies}

Optimization-based approaches explicitly plan footstep positions,
contact sequences, and timing, offering fine-grained control over
locomotion. Deits and Tedrake \cite{deits_footstep_2014} formulate
footstep placement as a mixed-integer quadratic program over convex
terrain regions, enabling robust bipedal foothold selection. However,
their method does not optimize contact timing and is limited to
sequential foot placement, restricting agility and generalization to
more complex contact sequences.

Extensions to this framework incorporate contact scheduling into the
optimization itself. Winkler et al. \cite{winkler_gait_2018} fully
optimize contact duration and ordering, allowing multiple steps to be
planned simultaneously rather than sequentially. While this increases
flexibility, the resulting computation is too slow for real-time
control. Similarly, Aceituno-Cabezas et al.
\cite{aceituno_simultaneous_2019} integrate footstep positions and
gait timings into a mixed-integer convex program based on centroidal
dynamics, achieving simultaneous contact and motion planning. Taouil
et al. \cite{taouil_non-gaited_2025} further explore non-gaited
footstep planning via Monte Carlo Tree Search (MCTS), generating
dynamic multi-step sequences in real time, though computation remains
high due to the branching factor. Akizhanov et al.
\cite{akizhanov_learning_2024} improve MCTS efficiency using a
learned classifier to prune contact configurations and a “target
adjustment network” to compensate for low-level control errors, but
the approach is still computationally intensive.

These optimization-based methods provide precise, globally consistent
locomotion plans, but high computational costs and limited real-time
applicability restrict their use in dynamic, non-gaited scenarios.
This motivates approaches that combine optimization principles with
learning-based or greedy strategies to achieve fast, terrain-adaptive
footstep planning.
