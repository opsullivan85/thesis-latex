\section{Synthesis}
\label{sec:background-synthesis}

The preceding review of quadruped locomotion control, encompassing
traditional optimization-based methods, end-to-end neural networks,
and hybrid control schemes, reveals a fundamental trade-off: control
theory guarantees and robustness versus computational efficiency and
adaptability to unstructured, non-gaited motion. Traditional methods
provide stability but suffer from high computational cost,
particularly when attempting to solve the mixed-integer programming
problem inherent in non-gaited contact planning
\cite{winkler_gait_2018, taouil_non-gaited_2024}. Conversely,
end-to-end learning approaches achieve impressive dynamic, acyclic
behaviors but lack formal guarantees and generalizability beyond
their training distribution \cite{zhang_learning_2023}.

Hybrid control architectures have emerged as the promising middle
ground, integrating learned modules within model-based frameworks to
balance agility and stability. However, existing hybrid footstep
planners, such as ContactNet \cite{bratta_contactnet_2024}, are often
constrained by enforcing sequential, one-leg-at-a-time motion or
relying on fixed gait timing. These constraints fundamentally limit
the robot's ability to execute the fully dynamic, acyclic locomotion
required for optimal performance in complex terrain.

This thesis addresses the identified gap by proposing a novel hybrid
architecture built upon the efficiency of greedy, NN-based selection
while overcoming the single-leg and fixed-timing limitations of prior
work. The GaitNet planner is designed to simultaneously
evaluate and select footstep candidates for multiple legs with
dynamic swing durations. By delegating the computationally intensive,
discrete decision-making of contact selection to this neural network
and feeding the resulting plans into a robust, model-based
controller, we aim to achieve a system that combines the agility and
non-gaited expressiveness of learning-based approaches with the
stability and physical consistency of traditional control frameworks.
This synthesis validates the necessity of a new hybrid approach
capable of supporting dynamic, multi-leg contact planning for robust
quadruped locomotion.
