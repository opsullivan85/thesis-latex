\section{Final Remarks}
\label{sec:conclusions-final-remarks}

\begin{outline}
  Conclude with a strong, definitive statement about your work's
  contribution to the field.
\end{outline}

This thesis set out to explore whether a hybrid control pipeline,
combining a greedy neural network-based planner with a model-based
controller, could generate dynamic and acyclic gaits for quadruped
locomotion. Through the design, training, and evaluation of
\textit{GaitNet}, this work demonstrated that such an approach is not
only feasible but can yield robust and adaptive locomotion behaviors
in challenging simulated environments. The results validate the
central hypothesis that hybrid architectures can bridge the gap
between the adaptability of learning-based methods and the
reliability of model-based control.

Beyond performance metrics, the findings carry broader implications
for the design of locomotion systems. The success of the greedy
planning strategy highlights the potential of lightweight,
inference-efficient neural networks for real-time decision-making in
high-dimensional control problems. Meanwhile, the observed
limitations and ablation results underscore the importance of
balanced modularity: while perception and candidate generation can
benefit from structured learning, value estimation and motion
selection appear to thrive under end-to-end reinforcement learning
paradigms that capture full dynamical context.

Equally important are the insights into learning dynamics and
representation. The discovery that removing pre-computed cost terms
improves performance challenges common assumptions about heuristic
guidance in reinforcement learning. This suggests that hybrid
controllers should not merely layer learned modules on top of
existing heuristics, but rather allow networks to develop their own
internal representations of value and risk through exploration. As
such, the future of legged locomotion control may depend less on
handcrafted structure and more on architectures that enable learned
coordination within well-defined safety boundaries.

In closing, this work contributes to the growing evidence that
data-driven methods can coexist effectively with analytical control
frameworks, each compensating for the other's weaknesses.
\textit{GaitNet} provides a concrete example of how this integration
can produce dynamic, adaptive, and interpretable motion
strategies—paving the way for future quadruped systems capable of
operating reliably in unstructured, real-world environments.
Continued efforts toward real-hardware validation, richer
environments, and predictive planning will further advance the vision
of truly agile, autonomous legged locomotion.
