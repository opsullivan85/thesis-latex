\section{Summary of Findings}
\label{sec:conclusions-summary-of-findings}

\begin{outline}
  Reiterate your key findings and how they support your hypothesis.
\end{outline}

This thesis presents GaitNet, a novel neural network-based greedy
planner for generating dynamic, acyclic gaits in quadruped robots
operating in challenging environments. Through a hierarchical hybrid
control architecture that combines learned footstep evaluation with
reinforcement learning-based gait selection, this work demonstrates
that efficient, terrain-adaptive locomotion can be achieved without
relying on predefined gait patterns.

GaitNet successfully generates non-periodic, dynamically stable gaits
that adapt to varying terrain conditions and commanded velocities.
Unlike traditional approaches that rely on fixed gait patterns (trot,
walk, etc.), the system learns to coordinate multi-leg motions with
variable timing, moving up to two legs simultaneously when conditions
permit. The generated swing schedules
(\autoref{fig:data-swing-schedule}) demonstrate true acyclic
behavior, with the robot adjusting its gait pattern in response to
both terrain difficulty and commanded velocity without following
repetitive cycles.

Comparative evaluation against a single leg motion planner baseline
reveals substantial performance improvements. GaitNet achieves a
\resGaitNetSurvivalRate{} survival rate across diverse terrain difficulties and
commanded velocities, compared to \resBaselineSurvivalRate{} for the
baseline method (\autoref{fig:data-experiments-survival-curr}
and~\autoref{fig:data-experiments-baseline-method}). This improvement
is particularly pronounced at higher speeds and on more challenging
terrain, demonstrating the value of dynamic gait coordination. The
baseline method's performance degrades rapidly above
\expLowestSpeed{} commanded velocity, while GaitNet maintains robust
performance up to 0.15\,m/s.

The footstep evaluation network, adapted from ContactNet
\cite{bratta_contactnet_2024}, successfully generates high-quality
footstep candidates for the gait selection process. The network
demonstrates strong generalization across diverse robot states,
accurately identifying low-cost foothold regions even in challenging
configurations (\autoref{fig:data-cn-typical-comparison}
and~\autoref{fig:data-cn-challenging-comparison}). While not perfect
in all scenarios, the network fulfills its role as a candidate
generator, providing sufficient diversity for downstream selection.

GaitNet learns to modulate swing durations based on environmental
conditions, with mean swing duration of
\resGaitNetMeanSwingDuration{} and standard deviation of
\resGaitNetSTDSwingDuration{} during evaluation. The system generally
selects swing durations between 0.225\,s and 0.25\,s, but
occasionally selects shorter swing durations when needed. However,
the swing duration ablation study reveals that this feature provides
only marginal performance benefits (2.2\% difference in survival
rate) in the tested environments, suggesting either that the current
scenarios do not fully exploit this capability or that the reward
structure does not adequately incentivize optimal duration selection.

Perhaps the most surprising finding emerges from the action cost
ablation study, which demonstrates that removing the footstep
candidate cost from GaitNet's input actually improves performance.
Cost-Ablated-GaitNet achieves a \resCostAblatedSurvivalRate{}
survival rate, representing an 8.3 percentage point improvement over
the standard formulation. The correlation analysis
(\autoref{fig:data-action-cost-correlation}) shows that while GaitNet
initially relies heavily on the provided costs, this dependence
decreases during training as the network learns to prioritize other
state features.

Cost-Ablated-GaitNet exhibits naturally more efficient behavior,
taking fewer redundant steps and maintaining a lower overall cadence
(\autoref{fig:data-action-cost-ablation-steps-per-second}) without
explicit reward shaping for these characteristics. This emergent
efficiency suggests that when freed from pre-computed heuristic
constraints, the reinforcement learning process discovers more
globally optimal locomotion strategies. The comparison of swing
schedules (\autoref{fig:data-action-cost-ablation-swing-duration})
illustrates this efficiency, with fewer instances of the same foot
being repositioned multiple times in succession.

The greedy planning approach enables real-time operation suitable for
online control. Unlike MCTS-based or full trajectory optimization
methods that require substantial computation for each decision,
GaitNet performs a single forward pass through a relatively compact
neural network to evaluate and rank discrete action candidates. This
computational efficiency is achieved while maintaining robust performance.

The hierarchical design successfully bridges learning-based and
model-based control paradigms. By decomposing the problem into
footstep candidate generation and gait selection, the architecture
provides interpretability and safety through explicit action
filtering while maintaining the adaptability of learned policies. The
strict constraints on possible actions prevent invalid or unsafe
motions while allowing sufficient flexibility for dynamic,
terrain-adaptive behavior.

In summary, this work validates the hypothesis that a greedy, neural
network-based planner integrated within a hybrid control framework
can generate dynamic, acyclic gaits for robust quadruped locomotion.
The findings provide valuable insights into the design of hybrid
locomotion controllers, particularly regarding the role of learned
heuristics and the balance between constraint and flexibility in
action selection. While absolute performance leaves room for
improvement, the demonstrated capabilities and unexpected insights
from ablation studies establish a foundation for future development
of adaptive legged locomotion systems.
