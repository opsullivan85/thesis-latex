\section{Future Work}

\begin{outline}
  Suggest potential avenues for future research, such as extending
  the method to real-world deployment, integrating different sensor
  modalities, or exploring multi-robot collaboration.
\end{outline}

\begin{itemize}
  \item Real-World Deployment

  \item Swing re-planning
    \begin{itemize}
      \item Currently, GaitNet doesn't re-plan at all during swings.
      \item This would be trivial to add to GaitNet, but would require
        a more robust low level controller.
    \end{itemize}

  \item Long horizon planning
    \begin{itemize}
      \item Currently GaitNet is only trained to output actions
        for a single point in time.
      \item The current implementation causes issues with the MPC
        not accurately knowing the future contact state. It looks
        like after every swing the robot will return to a nominal stance.
      \item This does not create significant issues at the low speeds
        being explored in this work, but would be problematic at faster speeds.
    \end{itemize}

  \item GPU accelerated MPC
    \begin{itemize}
      \item Currently the MPC is run on the CPU, which seriously limits
        the speed of RL learning.
      \item Some works explore GPU accelerated MPCs, which could be used
        to greatly speed up training.
    \end{itemize}

  \item Simultaneous actions
    \begin{itemize}
      \item Currently GaitNet only selects one action at a time.
      \item A method of selecting the two best actions at once was
        briefly explored, but was abandoned due to poor learning,
        and implementation challenges.
      \item specifically, I don't think I was able to get the gradients
        to flow properly through the two action selections.
    \end{itemize}

  \item Sampler improvements
    \begin{itemize}
      \item Currently GaitNet relies on a sampling based method
        to select actions.
      \item It should be possible to directly
        search the action space. This could be done with
        projected gradient ascent to handle the sparse action space,
        or some sort of diffusion?
      \item This would increase the quality of solutions at the
        cost of compute time.
      \item Unsure of how this would work with learning.
      \item Could also train candidate selection network in
        tandem with GaitNet to improve sampling quality.
    \end{itemize}
\end{itemize}
