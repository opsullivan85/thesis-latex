\section{Future Work}

Several avenues exist to extend and improve the current framework.
First, real-world deployment remains a crucial next step. Testing
GaitNet on physical hardware would expose the system to real-world
dynamics, sensor noise, and hardware constraints, providing valuable
insights for further refinement.

Swing re-planning is another potential improvement. Currently,
GaitNet does not re-plan during the swing phase. Incorporating
real-time re-planning would allow the robot to adjust to unexpected
disturbances or terrain changes, though this would require a more
robust low-level controller capable of accurately tracking the
modified footstep trajectories.

Long-horizon planning also presents an opportunity for enhancement.
GaitNet is presently trained to generate actions for a single point
in time, which limits the MPC's ability to predict future contact
states accurately; the MPC expects the robot to revert to a
nominal stance after each swing. While this is acceptable at low
speeds, it could become problematic at higher velocities, where
predictive planning would improve stability and performance.

Currently, the MPC runs on the CPU, limiting the speed of
reinforcement learning. Leveraging GPU-accelerated MPCs, as explored
in recent works \cite{todo}, could significantly accelerate training.

GaitNet also selects only one action at a time, leading to slightly
staggered swing start times. Initial attempts to
select multiple simultaneous actions encountered difficulties with
gradient flow and learning stability. Developing a reliable method
for multi-action selection could enable more fluid and dynamic gait
patterns, though it would require careful network and training design.

Finally, improvements to the action candidate sampler could enhance
GaitNet's performance. At present, action selection relies on
sampling-based methods, which may not always identify the
highest-quality actions. Directly searching the action space using
techniques such as projected gradient \cite{todo} ascent or other optimization
strategies could improve solution quality at the expense of
additional computation. Another possibility is to train a candidate
selection network jointly with GaitNet to improve the diversity and
relevance of sampled actions.
