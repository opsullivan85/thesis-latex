\section{Limitations}
\label{sec:limitations}

While the results presented in this work are promising, several
limitations should be acknowledged. First, the proposed framework has
only been validated in simulation. Real-world deployment may
introduce additional challenges, such as unmodeled dynamics, sensor
noise, and hardware constraints, which could affect performance.
Second, GaitNet operates in a greedy, no-lookahead fashion, limiting
its effectiveness at higher speeds or in highly dynamic environments
where predictive planning could improve stability and efficiency.
Additionally, the system's performance is constrained by the accuracy
of the low-level controller in executing footstep placements. Errors
in tracking or positioning can degrade the overall effectiveness of
the generated gaits. Finally, GaitNet's performance is inherently
tied to the quality of footstep candidates produced by the footstep
evaluation network; suboptimal candidate sampling may restrict both
the diversity and effectiveness of the generated actions.

These limitations highlight clear directions for future research,
including real-world testing on physical hardware, the incorporation
of predictive planning strategies, and improvements to both candidate
generation and low-level control precision.
