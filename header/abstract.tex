\begin{abstract}

  Quadruped locomotion across unstructured terrain remains a
  longstanding challenge in robotics, requiring controllers that can
  adapt dynamically to complex and unpredictable environments.
  Traditional approaches rely on model predictive control or
  pre-defined gait schedules, offering stability but limited
  adaptability. Conversely, end-to-end learning methods enable agile
  and versatile behaviors but often sacrifice interpretability,
  safety, and real-time feasibility. This thesis introduces
  \textit{GaitNet}, a hybrid control framework that  integrates a
  greedy, neural network-based gait planner with a traditional
  model-based controller to generate dynamic, acyclic locomotion for
  quadruped robots.

  The proposed system consists of two primary components: a
  \textit{Footstep Evaluation Network} that  learns terrain-aware
  footstep cost maps, and \textit{GaitNet}, a reinforcement
  learning-based gait selector that ranks and executes feasible
  footstep actions in real time. Together, these modules enable a
  balance between the adaptability of learning-based control and the
  robustness of analytical motion planning. Trained in NVIDIA Isaac
  Lab using parallel GPU simulation, the system is evaluated across a
  range of terrain difficulties and commanded velocities.

  Experimental results show that \textit{GaitNet} achieves a
  \resGaitNetSurvivalRate   mean survival rate in challenging
  terrain, outperforming a single-leg motion planner baseline by more
  than a factor of two. Ablation studies further reveal that dynamic
  swing duration offers limited benefit in static environments, while
  removing pre-computed footstep costs leads to improved performance
  and more efficient gait patterns. These findings highlight the
  advantages of direct policy learning for coordinated, multi-leg
  motion without overreliance on heuristic priors.

  Overall, this work demonstrates that greedy, neural network-based
  planning can produce dynamic, non-gaited quadruped motion with \high
  computational efficiency and strong generalization to complex
  terrain. The presented framework bridges the gap between reactive
  learning and structured control, providing a foundation for future
  research toward fully autonomous, terrain-adaptive legged
  locomotion in real-world settings.

\end{abstract}
