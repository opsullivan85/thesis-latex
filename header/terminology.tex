\chapter*{Terminology}

Within legged robotics, there are many conflicting or ambiguous terms. For clarity, the exact meanings of specific terms used in this paper are described below.


\begin{table}[h!]
    \centering
    \begin{tabular}{l p{10cm}}
         Term & Definition\\
         \hline
         Footstep planning & The process of selecting when and where to step. \vspace{0.5em}\\
         Greedy & In the context of footstep planning, a planner which only provides instantaneous actions. \vspace{0.5em}\\
         Dynamic locomotion/gait & The opposite of quasi-static locomotion; where balance is maintained through active control and dynamic interactions, not slow and stable placements. Sometimes as an adjective, dynamic (as in the title). \vspace{0.5em}\\
         Acyclic gait & Sometimes "free-gait" or "non-gaited". Refers to a footstep plan which cannot be represented with phase cycles or transitions between gait primitives. The opposite of gaited or gait-based locomotion. \vspace{0.5em}\\
    \end{tabular}
\end{table}