\section{Motivation and Vision}

\begin{outline}
  Discuss the challenges of quadruped locomotion in unstructured
  environments and the limitations of traditional gaited methods.
\end{outline}

Quadruped control pipelines fall into many different categories, but
most can be broadly classified as either traditional, neural
network-based, or hybrid methods.

Traditional methods often rely on
an MPC to perform optimization and lower level controllers to track the
desired motions. Many parts of quadruped locomotion are smooth
and can be trivially optimized, such as joint torques or ground
reaction forces. These methods become more impractical when
attempting to optimize over the discrete contact state of the robot.
Typically, these methods use a pre-defined gait sequence or integrate
the discrete contacts into a more complicated optimization problem.
This trade-off often works well, but ultimately limits the
capabilities of the robot in certain terrains.

Neural network-based methods have shown promise in recent years,
especially with the rise of reinforcement learning techniques.
Often, these methods replace the MPC in a traditional pipeline,
taking in state information and directly outputting joint commands.
These methods implicitly solve the discrete contact problem, with
the network learning to optimize the whole mixed-integer problem.
In exchange for this flexibility, these methods loose the guarantees
associated with traditional methods, require large amounts of data
to train, and often generalize poorly in new environments.

Falling in the middle of these two approaches are hybrid methods,
which attempt to combine the best of both worlds. This will be the
focus of this thesis. These methods often
use neural networks to solve certain aspects of the problem while
a traditional controller performs the tracking. This allows the network
to focus on the more difficult parts of the problem, such as contact
planning, while still leveraging the stability and reliability of
traditional controllers.
