\section{Motivation and Vision}
\label{sec:motivation-and-vision}

Quadruped control pipelines can be categorized into several
approaches, most of which can be broadly classified as traditional,
neural network-based, or hybrid methods.

Traditional methods often rely on MPC for optimization, combined with
lower-level controllers to track the desired motions
\cite{Kim2019highly-dynamic}. Many aspects of quadruped
locomotion—such as joint torques and ground reaction forces—are
smooth and can be efficiently optimized \cite{Wensing2022Nov}.
However, these methods become less practical when optimizing over the
robot's discrete contact states
\cite{Geisert2019contact-planning}. Typically, they employ
pre-defined gait sequences
\cite{Chai2022survey, Fan2024survey} or embed discrete contact
decisions into a more complex optimization problem
\cite{winkler_gait_2018}. This trade-off performs well in general
settings, but ultimately limits the unique capabilities of quadruped robots.

Neural network-based methods have gained traction in recent years,
particularly with the advancement of reinforcement learning
techniques. These approaches often replace the MPC component of
traditional pipelines, directly mapping state information to joint
commands \cite{Gurram2024survey}. In doing so, they implicitly
address the discrete contact problem, allowing the network to learn
the mixed-integer optimization implicitly \cite{Wensing2022Nov}.
Although this provides greater flexibility, such methods sacrifice
the formal guarantees of traditional control, demand extensive
training data, and frequently exhibit poor generalization to novel
environments \cite{Gurram2024survey}.

Hybrid methods, which combine elements of both approaches, form the
focus of this thesis. These methods leverage neural networks to
address specific challenges—such as contact or foothold
planning—while relying on traditional controllers for carrying out
higher level tasks \cite{Bao2024survey, Wang2022Oct}. This synergy
enables the neural network to concentrate on the more complex and
non-smooth aspects of locomotion, while maintaining the stability and
robustness offered by model-based control frameworks.
