\section{Motivation and Vision}
\label{sec:introduction-motivation-and-vision}

Legged robots hold significant promise for enabling mobility in
environments that are inaccessible to wheeled and tracked systems.
Their ability to traverse uneven, unstructured, and unpredictable
terrains—such as disaster zones, staircases, dense vegetation, and
rocky landscapes—positions them as key enablers for future
exploration, inspection, and rescue missions.

Realizing this potential requires a robot to continuously and
intelligently decide where and when to place its feet—a process known
as \textit{contact planning}. Unlike periodic gait patterns suitable
for flat or predictable surfaces, real-world locomotion often demands
dynamic, acyclic, and adaptive footstep sequences. Even minor errors
in contact selection can lead to instability or failure, whereas
well-chosen footholds enable efficient, agile, and robust movement.

The generation of such contact plans in real time remains a
formidable challenge. It requires the system to perceive its
surroundings, anticipate the outcomes of possible actions, and select
feasible contact points within tight temporal constraints.
\hl{Existing methods} often trade off between deliberative,
computationally intensive planning and fast, reactive control that
lacks stability or generalization.

% discuss hybrid approaches before this. Unclear that hybrid
% approaches exist at this point. seems like I'm about to invent them

\hl{The vision of this research is to develop a control architecture that
bridges this divide}. Specifically, this work aims to design a system
capable of producing non-gaited, dynamic footstep plans that retain
both the responsiveness of learning-based approaches and the
reliability of model-based control. The objective is to provide
quadruped robots with the decision-making agility necessary for
navigating complex terrains while maintaining the stability and
robustness demanded by real-world operation.

Through this approach, the thesis seeks to contribute toward a new
class of hybrid locomotion frameworks that enable real-time,
adaptive, and physically consistent quadruped motion—closing the gap
between high-level perception and low-level control in legged robotics.
