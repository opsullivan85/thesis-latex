\section{Research Gap}
\label{sec:introduction-research-gap}

Quadruped control pipelines can be categorized into several
approaches, many of which can be broadly classified as traditional,
neural network-based, or hybrid methods.

Traditional approaches typically rely on model predictive control
(MPC) frameworks for motion optimization, supported by lower-level
controllers to track desired trajectories
\cite{Kim2019highly-dynamic}. Many aspects of quadruped
locomotion—such as joint torques and ground reaction forces—are
smooth and well-suited to continuous optimization techniques
\cite{Wensing2022Nov}. However, these methods become less tractable
when discrete contact states are introduced into the optimization
problem \cite{Geisert2019contact-planning}. To manage this
complexity, most implementations either assume pre-defined gait
sequences \cite{Chai2022survey, Fan2024survey} or encode discrete
contact decisions within larger optimization structures
\cite{winkler_gait_2018}. While effective in structured environments,
these strategies constrain the robot's ability to exploit the full
versatility of legged locomotion.

Neural network-based methods have emerged as an alternative,
particularly with advances in deep reinforcement learning. These
approaches often replace the MPC component of traditional control
pipelines, learning to directly map observed states to joint torques
or motion commands \cite{Gurram2024survey}. In doing so, they
inherently address the mixed-integer nature of contact planning,
allowing the network to learn both continuous and discrete aspects of
locomotion simultaneously \cite{Wensing2022Nov}. Despite their
flexibility, such methods generally lack formal guarantees of
stability, require substantial training data, and often struggle to
generalize beyond their training distribution \cite{Gurram2024survey}.

Hybrid approaches, which integrate learning-based modules within
model-based control frameworks, have recently gained attention as a
potential middle ground. In these systems, neural networks are
typically employed to solve specific subproblems—such as contact
selection or foothold prediction—while the overall motion execution
remains governed by a traditional controller \cite{Bao2024survey,
Wang2022Oct}. This division of responsibilities enables learning to
focus on the non-smooth or combinatorial components of locomotion,
while preserving the interpretability, safety, and robustness
inherent to model-based systems.

Although hybrid control schemes have demonstrated promise, the
challenge of contact and footstep planning continues to limit their
performance. Traditional optimization-based planners struggle with
the discrete nature of contact transitions, resulting in high
computational costs that preclude real-time operation
\cite{winkler_gait_2018}. Alternatively, many rely on pre-specified
gait patterns \cite{xie_glide_2023, grandia_perceptive_2022,
lee_learning_2020, villarreal_fast_2019}, constraining adaptability
to irregular or unpredictable terrain. Even more advanced approaches,
such as MCTS-based planners, have achieved impressive dynamic
behaviors but remain computationally demanding
\cite{amatucci_monte_2022, taouil_non-gaited_2025}.

Despite recent progress, a clear research gap remains. Current hybrid
methods do not yet achieve the balance between adaptability and
stability that would enable fully dynamic, acyclic locomotion in
complex environments. In particular, there is a lack of systems that
can match the contact planning agility of neural network-based
approaches while retaining the formal guarantees and reliability of
traditional control frameworks \cite{Meng2023Mar, Wensing2022Nov}.
Recent work, such as ContactNet \cite{bratta_contactnet_2024}, has
demonstrated the potential of machine learning for footstep planning;
however, its design—restricted to single-leg motions with fixed
timing—still falls short of the dynamic, non-gaited behaviors
exhibited by end-to-end learning systems.

This motivates a central research question:

\begin{emphasis}
  Can a hybrid control pipeline, which integrates a greedy,
  neural-network-based planner with a traditional   model-based
  controller, generate dynamic, acyclic gaits for robust   quadruped
  locomotion in challenging environments?
\end{emphasis}
