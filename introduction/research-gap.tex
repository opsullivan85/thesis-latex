\section{Research Gap}

Traditional methods have continually faced challenges in footstep
planning. Some approaches attempt to incorporate the discrete aspects
of contact planning directly into the optimization problem,
but this introduces significant computational complexity, making
real-time solutions difficult \cite{winkler_gait_2018}. Other methods
rely on pre-defined gait
sequences
\cite{xie_glide_2023, grandia_perceptive_2022, lee_learning_2020,
villarreal_fast_2019},
which constrain the robot's
ability to fully exploit the versatility of quadruped locomotion.
Although substantial progress has been made in MCTS-based footstep
planners for classical controllers, these approaches remain
computationally demanding \cite{amatucci_monte_2022, taouil_non-gaited_2025}.

Despite the advantages and recent advancements in hybrid control
methods, a notable gap persists in the literature. Specifically,
there is a lack of hybrid approaches capable of achieving the robust
contact planning performance of neural network-based systems while
preserving the stability and reliability of traditional control
frameworks. Recent work (Contactnet) has demonstrated promise in
generating footstep plans through machine learning; however, this
method is constrained to moving one leg at a time for a pre-defined
duration—falling short of the fully dynamic and acyclic gaits
observed in end-to-end learning systems.

This motivates a central research question:

\begin{emphasis}
  Can a machine learning-based planner generate dynamic, acyclic
  gaits for quadruped robots operating in challenging environments?
\end{emphasis}
