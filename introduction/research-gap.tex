\section{Research Gap}

\begin{outline}
  Explain the need for more adaptive, non-gaited approaches and the
  specific problem you are solving: balancing computational
  efficiency with dynamic behaviors for real-time control.
\end{outline}

% Quadruped control pipelines fall into many different categories, but
% most can be broadly classified as either traditional, neural
% network-based, or hybrid methods.

% Traditional methods often rely on
% an MPC to perform optimization and lower level controllers to track the
% desired motions. Many parts of quadruped locomotion are smooth
% and can be trivially optimized, such as joint torques or ground
% reaction forces. These methods become more impractical when
% attempting to optimize over the discrete contact state of the robot.
% Typically, these methods use a pre-defined gait sequence or integrate
% the discrete contacts into a more complicated optimization problem.
% This trade-off often works well, but ultimately limits the
% capabilities of the robot in certain terrains.

% Neural network-based methods have shown promise in recent years,
% especially with the rise of reinforcement learning techniques.
% Often, these methods replace the MPC in a traditional pipeline,
% taking in state information and directly outputting joint commands.
% These methods implicitly solve the discrete contact problem, with
% the network learning to optimize the whole mixed-integer problem.
% In exchange for this flexibility, these methods loose the guarantees
% associated with traditional methods, require large amounts of data
% to train, and often generalize poorly in new environments.

% Falling in the middle of these two approaches are hybrid methods,
% which attempt to combine the best of both worlds. These methods often
% use neural networks to solve certain aspects of the problem while
% a traditional controller performs the tracking. This allows the network
% to focus on the more difficult parts of the problem, such as contact
% planning, while still leveraging the stability and reliability of
% traditional controllers.

Traditional methods have continually struggled with footstep planning.
Some approaches attempt to integrate the discrete aspects of contact
planning into the optimization problem (TOWR), but this creates
significant computational challenges when trying to solve in real-time.
Other approaches rely on pre-defined gait sequences (Raibert-style gaits, etc.),
but these approaches do not allow for full utilization of the quadruped's
unique capabilities. There has been significant groundwork made in MCTS-based
footstep planners for use with classical controllers, but these methods still
prove computationally intensive.

Despite the advantages and recent advancements of the hybrid methods,
there is still a significant gap in the literature when it comes to
hybrid control methods which can match the robust contact planning
of neural network-based methods while still maintaining the
reliability and stability of traditional methods. Recently, (Contactnet)
showed potential for generating footstep plans using a machine learning
approach, but this method is limited to moving one leg at a time, and
for a pre-defined duration\textemdash falling short of the fully
dynamic and acyclic gaits observed in end-to-end learning methods.

All of this begs the question:

\begin{emphasis}
  Can a machine learning planner generate dynamic, acyclic gaits for
  quadruped robots in challenging environments?
\end{emphasis}
