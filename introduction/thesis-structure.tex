\section{Thesis Structure}
\label{sec:introduction-thesis-structure}

\begin{outline}
  Provide a brief roadmap of the remaining chapters.
\end{outline}

The remaining chapters are organized as follows.

Chapter 2 presents the background necessary to contextualize this
work, covering fundamental concepts in quadruped locomotion, gait
generation, and learning-based control methods. This chapter
establishes the theoretical foundations on which the proposed approach is built.

Chapter 3 describes the methodology, detailing the design and
implementation of the CNN-based greedy planner, GaitNet, including
the underlying network architecture and training pipeline. It
explains how this module integrates with a model-based control
framework to enable dynamic footstep planning.

Chapter 4 reports the experimental results and evaluates the
performance of the proposed planner across a range of challenging
scenarios. This chapter also discusses the strengths and limitations
of the method, examining how the system behaves under diverse conditions.

Chapter 5 concludes the thesis by summarizing the key findings and
reflecting on the broader implications for quadruped locomotion
research. It also identifies potential directions for future work,
particularly avenues for improving adaptability, stability, and
real-world deployment.

% \begin{itemize} % \item \textbf{Chapter 2: Background} - Covers
% foundational concepts % in quadruped locomotion, gait generation,
% and ML/RL % techniques relevant to this work. % \item
% \textbf{Chapter 3: Methodology} - Details the design and %
% implementation of the CNN-based greedy planner, including network %
% architecture and training procedures. % \item \textbf{Chapter 4:
% Results and Discussion} - Presents % experimental results,
% evaluates the performance of the proposed % planner, and discusses
% its strengths and limitations. % \item \textbf{Chapter 5:
% Conclusions and Future Work} - Summarizes % the key findings of the
% thesis and outlines potential directions % for future research. %
% \end{itemize}
