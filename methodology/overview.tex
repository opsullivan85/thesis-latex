\section{System Overview}
\label{sec:methodology-system-overview}
\begin{outline}
  Present a high-level overview of your proposed control framework,
  likely including a block diagram (similar to Figure 1 in the
  proposal) and a description of each component.
\end{outline}
This work presents a control framework for quadruped robots capable
of generating dynamic, acyclic gaits in challenging environments. The
framework employs a hierarchical architecture that processes robot
state and terrain data to produce footstep commands for the robot controller.

The first stage of the framework comprises a footstep evaluation
network, similar to ContactNet from \cite{bratta_contactnet_2024},
which estimates candidate footsteps $\mathbf{f_c}$. The second stage,
a gait generation network (GaitNet), receives these candidates
$\mathbf{f_c}$, ranks them, and outputs the optimal footstep action
$\mathbf{f_a}$ to the robot controller. The robot controller is the
MIT Mini-Cheetah Controller \cite{di_mini_cheetah_2020}, implemented
in python by \cite{zhuang_rl_mpc_locomotion_2025}.

This hierarchical design enables strict control over the range of
possible actions the robot can execute. Between the two stages,
candidate actions are filtered to ensure that only valid, prescribed
movements are permitted.

\begin{todo}
  Update \autoref{fig:diagram-control-system} with images of the
  experimental environment. Also update terminology for different
  control blocks and connections between them.
\end{todo}

\begin{figure}[H]
  \centering
  \includegraphics[width=1.0\linewidth]{images/diagrams/control-system.png}
  \caption{Block diagram of the proposed framework. Novel aspects
    shown in blue. The user defines an input direction
    $\mathbf{v^{usr}}$, which the \textit{Footstep Evaluation
    Network} (\autoref{sec:methodology-footstep-evaluation-network}), based on
    \textit{ContactNet}
    \cite{bratta_contactnet_2024}, uses along with the robot state
    $\mathbf{x_c}$ and the current foot positions $\mathbf{p_f}$ to
    generate a footstep cost map. The footstep cost map is then
    processed
    (\autoref{subsec:methodology-contactnet-post-processing}) to
    produce a set of candidate footstep actions $\mathbf{f_c}$. These
    candidates are passed to the \textit{GaitNet}
    (\autoref{sec:methodology-gaitnet}), which selects
    one\protect\footnotemark action $\mathbf{f_a}$ to send to the
    robot controller. The \textit{MIT Mini-Cheetah Controller}
    \cite{di_mini_cheetah_2020} receives $\mathbf{f_a}$ and executes
  the corresponding actions.}
  \label{fig:diagram-control-system}
\end{figure}

\footnotetext{As is explained in later sections, GaitNet only outputs
  one footstep action at a time. The two separate foot target locations
  shown in the GaitNet visual would be selected sequentially with a
  1/20s delay. This visual is meant to highlight the near simultaneous
nature of the footstep selection process.}
