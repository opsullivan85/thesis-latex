\section{Simulation Environment}
\label{sec:simulation-environment}

\begin{outline}
  Describe the physics-based simulation environment (IsaacLab) and
  the quadruped model used for development and testing.
\end{outline}

The simulation environment used for this project is NVIDIA Isaac Lab
\cite{mittal_orbit_2023}. This framework was selected for several
reasons. It is a modern platform with a Python interface designed
specifically for machine learning applications and GPU parallelism.
The use of GPU parallelism enables significantly faster simulation
and data collection, albeit at the cost of increased programming
complexity and reduced compatibility with older hardware.
Furthermore, the extensive collection of example and community
projects provides valuable references for implementing the simulation
features required in this work.

Although both simulation and learning processes are executed on the
GPU, the robot controllers operate on the CPU.
\autoref{fig:diagram-processing-flow} illustrates the overall
processing flow. The simulation environment runs entirely on the GPU,
where multiple robots are simulated in parallel. Meanwhile, the robot
MPCs execute on the CPU, with each controller running concurrently.
The CPU and GPU communicate at every simulation step to exchange
robot states and corresponding control actions.

\begin{todo}
  add in learning step
\end{todo}

\begin{figure}[H]
  \centering
  \includegraphics[width=0.5\linewidth]{images/diagrams/processing-flow.png}
  \caption{A block diagram showing the programming tasks computed on
    the CPU vs GPU. The full simulation is run in parallel using Nvidia
  Isaac Lab on the GPU, while the robot MPCs are run in parallel on the CPU.}
  \label{fig:diagram-processing-flow}
\end{figure}

NVIDIA Isaac Lab uses a declarative system of python dataclasses to
define the simulation environment. For this work, a custom
environment is used (\autoref{fig:figure-terrain-raycast}), including:

\begin{todo}
  update
\end{todo}

\begin{itemize}
  \item Unitree Go 1\textemdash Configured to use force control for each joint.
  \item Terrain raycasts\textemdash Measure the height of the terrain
    at each possible footstep location. This emulates the lidar
    processing steps of a full vision pipeline.
  \item Custom terrain\textemdash Planar terrain with sections
    missing on a grid pattern. The underlying grid is 8\,cm square,
    and     the void density is tuned to create a challenging but
    possible     environment. Multiple difficulty levels of the
    terrain are generated     for the robots to progress through as
    they learn.     The terrain is designed to mirror that used in
    \cite{bratta_contactnet_2024}
    through as they learn.
\end{itemize}

\begin{todo}
  update \autoref{fig:figure-terrain-raycast}
\end{todo}

\begin{figure}[H]
  \centering
  \includegraphics[width=0.75\linewidth]{images/figures/terrain-raycast.png}
  \caption{Image of a Unitree Go 1 navigating the terrain. Red
    spheres show the hit location of raycasts. Black regions show voids
  in the terrain.}
  \label{fig:figure-terrain-raycast}
\end{figure}
