\section{Action Cost Ablation Study}
\label{sec:results-action-cost-ablation-study}

In this section we present an ablation study to assess the impact of
the footstep candidate cost on GaitNet's performance. We compare two
models: the standard GaitNet formulation as described in
\autoref{sec:methodology-gaitnet}, and a Cost-Ablated-GaitNet trained with the
the footstep candidate ($\mathbf f_c$) zeroed out in the input. In
the GaitNet formulation described previously, we pass the estimated
heuristic cost of a footstep action into the model ($\mathbf
f_c^\prime$ in \autoref{fig:diagram-gaitnet-architecture}), that is,
the expected cost of taking the action if it were the only motion the
robot was performing. For this ablation study we replace that value
with a zero during the training and evaluation process to see how it
affects the network without changing the shape of the network at all.

During training, the GaitNet model quickly learns to correlate the
lower cost candidates with higher value logits. After this initial
learning phase, the model learns to refine its predictions based on
other features in the input. This is illustrated in
\autoref{fig:data-action-cost-correlation}.

The inspiration for this ablation study comes from observing that the
footstep candidate cost and GaitNet output logits become less
correlated as training progresses. This suggests that while the
footstep candidate cost is useful for initial learning, the model may
be locked into a local minimum where it relies too heavily on this feature.

\begin{figure}[H]
  \centering
  \includegraphics[width=\smallplotsize{}]{images/data/training/gaitnet/logit_cost_correlation_survival-curr.png}
  \caption{Correlation between negative footstep candidate cost and
    GaitNet output logits during training. Negative costs are used so
    that a costmap which perfectly captures optimal footstep locations
  has a correlation of +1.}
  \label{fig:data-action-cost-correlation}
\end{figure}

Now, we train the Cost-Ablated-GaitNet model, over-writing all
footstep candidate cost values with zero. The training process is
identical to that of the standard GaitNet model, ensuring a fair
comparison. The performance of both models is then evaluated using
the same metrics outlined in
\autoref{sec:results-baseline-comparison}.

\begin{figure}[H]
  \centering
  \includegraphics[width=\smallplotsize{}]{images/data/experiments/Gaitnet
  - footstep-cost - 77.7 - individual.png}
  \caption{Evaluation of Cost-Ablated-GaitNet
    across various terrain difficulties and commanded velocities.
    Overall survival rate of \resCostAblatedSurvivalRate{}. Mean
    success rate measured as the percentage of
    \expBenchmarkEpisodes{} episodes which completed
    \expBenchmarkDuration{} without terminating, under the
    termination conditions described in
  \autoref{sec:appendix-termination-functions}.}
  \label{fig:data-action-cost-ablation-comparison}
\end{figure}

\autoref{fig:data-action-cost-ablation-comparison} presents the
performance of the Cost-Ablated-GaitNet. The results indicates a
measurable performance over the standard GaitNet formulation. GaitNet
was able to achieve an overall survival rate of
\resGaitNetSurvivalRate{}
(\autoref{fig:data-experiments-survival-curr}), while the
Cost-Ablated-GaitNet achieved a survival rate of
\resCostAblatedSurvivalRate{}
(\autoref{fig:data-action-cost-ablation-comparison}).

Looking at the mean episode reward in
\autoref{fig:data-action-cost-ablation-reward}, we see a similar
trend. The Cost-Ablated-GaitNet achieves a higher mean episode reward
during training compared to the standard GaitNet. This does come at
the cost of initially slower learning, where the Cost-Ablated-GaitNet
trails the performance of the standard GaitNet for the first
\ $\sim$75 iterations.

\begin{figure}[H]
  \centering
  \includegraphics[width=0.5\textwidth]{images/data/training/gaitnet/mean_reward_survival-curr_vs_footstep-cost.png}
  \caption{Mean episode reward during training for GaitNet and
  Cost-Ablated-GaitNet.}
  \label{fig:data-action-cost-ablation-reward}
\end{figure}

Interestingly, the Cost-Ablated-GaitNet also demonstrates a slower
cadence during training, as shown in
\autoref{fig:data-action-cost-ablation-steps-per-second}.

\begin{figure}[H]
  \centering
  \includegraphics[width=0.5\textwidth]{images/data/training/gaitnet/steps_per_second_survival-curr_vs_footstep-cost.png}
  \caption{Comparison of steps per second during training for GaitNet
  and Cost-Ablated-GaitNet.}
  \label{fig:data-action-cost-ablation-steps-per-second}
\end{figure}

Looking into how this reduced cadence affects the swing schedule, we
see in \autoref{fig:data-action-cost-ablation-swing-duration} that
under the same test parameters (10\% terrain difficulty and 0.15\,m/s
commanded velocity), the Cost-Ablated-GaitNet takes less redundant
steps\textemdash when the same foot is moved multiple times in succession.

\begin{figure}[H]
  \centering
  \includegraphics[width=1\textwidth]{images/data/swing-schedule-cost-ablation-comparison-d0.1-v0.15.png}
  \caption{Comparison of swing schedules for GaitNet (top)
    and Cost-Ablated-GaitNet (bottom) at 10\% terrain difficulty with a
  commanded velocity of 0.15\,m/s.}
  \label{fig:data-action-cost-ablation-swing-duration}
\end{figure}

\subsection{Discussion}

The action cost ablation study provides perhaps the most significant
insight. Counterintuitively, removing the footstep candidate cost
from GaitNet's input improves performance from \resGaitNetSurvivalRate{} to
\resCostAblatedSurvivalRate{} survival rate. This result challenges the initial
assumption that providing the learned heuristic costs would
accelerate training and improve final performance. Several factors
likely contribute to this outcome. The correlation analysis
(\autoref{fig:data-action-cost-correlation}) shows that while GaitNet
initially relies heavily on the cost term (correlation near 0.8),
this dependence decreases during training, suggesting the network
learns to prioritize other state features. The inclusion of the cost
term may constrain exploration, biasing the network toward locally
optimal solutions that align with the footstep evaluation network's
heuristics but prevent discovery of globally better strategies.

Intuitively this makes sense, as the footstep cost term included in
the standard GaitNet model provides information about the robot
dynamics, learned from the Footstep Evaluation Network. Without this
information, the Cost-Ablated-GaitNet must learn the robot dynamics
from scratch, leading to slower initial learning. However, as noted,
this allows for more freedom to explore. Moreover,
Cost-Ablated-GaitNet exhibits a naturally slower cadence
(\autoref{fig:data-action-cost-ablation-steps-per-second}) and fewer
redundant footsteps
(\autoref{fig:data-action-cost-ablation-swing-duration}), indicating
more deliberate and efficient gait patterns. This behavior emerges
without different reward shaping, suggesting that the model better
captures the underlying dynamics when not constrained by pre-computed
costs. The slower initial learning observed in Cost-Ablated-GaitNet
represents a worthwhile trade-off for superior final performance.
