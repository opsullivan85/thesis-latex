\section{Action Cost Ablation Study}
\label{sec:results-action-cost-ablation-study}

In this section we present an ablation study to assess the impact of
the footstep candidate cost on GaitNet's performance. We compare two
models: the standard GaitNet formulation as described in
\autoref{sec:methodology-gaitnet}, and a Cost-Ablated-GaitNet trained without
the footstep candidate ($f_c$) cost in the input.

During training, the GaitNet model quickly learns to correlate the
lower cost candidates with higher value logits. After this initial
learning phase, the model learns to refine its predictions based on
other features in the input. This is illustrated in
\autoref{fig:data-action-cost-correlation}.

\begin{figure}[H]
  \centering
  \includegraphics[width=0.45\textwidth]{example-image-a}
  \caption{Correlation between negative footstep candidate cost and
  GaitNet output   logits during training.}
  \label{fig:data-action-cost-correlation}
\end{figure}

Now, we train the Cost-Ablated-GaitNet model, over-writing all
footstep candidate cost values with zero. The training process is
identical to that of the standard GaitNet model, ensuring a fair
comparison. The performance of both models is then evaluated using
the same metrics outlined in
\autoref{sec:results-baseline-comparison}.

\textit{RESULTS IN PROGRESS}

% \begin{figure}[H] %   \centering %
% \includegraphics[width=\textwidth]{example-image-a} %
% \caption{Evaluation of GaitNet vs. Cost-Ablated-GaitNet %
% across various terrain difficulties and commanded velocities. %
% Mean success rate measured as the percentage of 50 episodes %
% which completed 20\,s without terminating, under the termination %
%  conditions described in
% \autoref{sec:appendix-termination-functions}.} %
% \label{fig:data-action-cost-ablation-comparison} % \end{figure}

% \begin{todo} %   if the action cost and logits are very poorly
% correlated, then %   there is   possible improvement in the
% candidate action sampling method. % \end{todo}

% \autoref{fig:data-action-cost-ablation-comparison} presents the %
% performance comparison between the standard GaitNet and the %
% cost-ablated variant. The results indicates that there is minimal %
% difference in performance between the two models. However, looking
% at % the episode reward for living %
% (\autoref{fig:data-action-cost-ablation-reward}), the cost-ablated
% % model can be seen to learn more slowly than the standard GaitNet.
% % This indicates that the footstep candidate cost is useful during
% the % initial stages of training to help the model quickly learn
% the robot dynamics.

% \begin{figure}[H] %   \centering %
% \includegraphics[width=0.5\textwidth]{example-image-a} %
% \caption{Episode reward for living during training for GaitNet %
% and Cost-Ablated-GaitNet.} %
% \label{fig:data-action-cost-ablation-reward} % \end{figure}
