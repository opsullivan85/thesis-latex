\section{Action Cost Ablation Study}
\label{sec:results-action-cost-ablation-study}

In this section we present an ablation study to assess the impact of
the footstep candidate cost on GaitNet's performance. We compare two
models: the standard GaitNet formulation as described in
\autoref{sec:methodology-gaitnet}, and a Cost-Ablated-GaitNet trained with the
the footstep candidate ($f_c$) zeroed out in the input.

During training, the GaitNet model quickly learns to correlate the
lower cost candidates with higher value logits. After this initial
learning phase, the model learns to refine its predictions based on
other features in the input. This is illustrated in
\autoref{fig:data-action-cost-correlation}.

The inspiration for this ablation study comes from observing that the
footstep candidate cost and GaitNet output logits become less
correlated as training progresses. This suggests that while the
footstep candidate cost is useful for initial learning, the model may
be locked into a local minimum where it relies too heavily on this feature.

\begin{figure}[H]
  \centering
  \includegraphics[width=0.45\textwidth]{images/data/training/gaitnet/logit_cost_correlation_survival-curr.png}
  \caption{Correlation between negative footstep candidate cost and
    GaitNet output logits during training. Negative costs are used so
    that   a costmap which perfectly captures optimal footstep
  locations has a    correlation of +1.}
  \label{fig:data-action-cost-correlation}
\end{figure}

Now, we train the Cost-Ablated-GaitNet model, over-writing all
footstep candidate cost values with zero. The training process is
identical to that of the standard GaitNet model, ensuring a fair
comparison. The performance of both models is then evaluated using
the same metrics outlined in
\autoref{sec:results-baseline-comparison}.

\begin{figure}[H]
  \centering
  \includegraphics[width=0.45\textwidth]{images/data/experiments/Gaitnet
  - footstep-cost - 77.7 - individual.png}
  \caption{Evaluation of Cost-Ablated-GaitNet
    across various terrain difficulties and commanded velocities.
    Overall survival rate of 77.7\%.     Mean success rate measured
    as     the percentage of 50 episodes     which completed 20\,s
    without     terminating, under the termination   conditions described in
  \autoref{sec:appendix-termination-functions}.}
  \label{fig:data-action-cost-ablation-comparison}
\end{figure}

\autoref{fig:data-action-cost-ablation-comparison} presents the
performance of the Cost-Ablated-GaitNet. The results indicates  a
measurable performance over the standard GaitNet formulation. GaitNet
was able to achieve an overall survival rate of 69.4\%
(\autoref{fig:data-experiments-survival-curr}), while the
Cost-Ablated-GaitNet achieved a survival rate of 77.7\%
(\autoref{fig:data-action-cost-ablation-comparison}). This suggests
that including the footstep candidate cost is detrimental to the
training process.

Looking at the mean episode reward in
\autoref{fig:data-action-cost-ablation-reward}, we see a similar
trend. The Cost-Ablated-GaitNet achieves a higher mean episode reward
during training compared to the standard GaitNet. This does come at
the cost of initially slower learning, where the Cost-Ablated-GaitNet
trails the performance of the standard GaitNet for the first
\ $\sim$75 iterations. Intuitively this makes sense, as the footstep
cost term included in the standard GaitNet model provides information
about the robot dynamics, learned from the Footstep Evaluation
Network. Without this information, the Cost-Ablated-GaitNet must
learn the robot dynamics from scratch, leading to slower initial learning.

\begin{figure}[H]
  \centering
  \includegraphics[width=0.5\textwidth]{images/data/training/gaitnet/mean_reward_survival-curr_vs_footstep-cost.png}
  \caption{Mean episode reward during training for GaitNet and
  Cost-Ablated-GaitNet.}
  \label{fig:data-action-cost-ablation-reward}
\end{figure}

Interestingly, the Cost-Ablated-GaitNet also demonstrates a slower
cadence during training, as shown in
\autoref{fig:data-action-cost-ablation-steps-per-second}. With the
Cost-Ablated-GaitNet naturally taking fewer steps, it is likely that
a more optimal reward function could be designed. As is described in
\autoref{sec:appendix-reward-function-analysis}, the current reward
is designed to avoid un-necessary footsteps, which could be further
limiting the performance of this Cost-Ablated-GaitNet model.

\begin{figure}[H]
  \centering
  \includegraphics[width=0.5\textwidth]{images/data/training/gaitnet/steps_per_second_survival-curr_vs_footstep-cost.png}
  \caption{Comparison of steps per second during training for GaitNet
  and Cost-Ablated-GaitNet.}
  \label{fig:data-action-cost-ablation-steps-per-second}
\end{figure}

Looking into how this reduced cadence affects the swing schedule, we
see in \autoref{fig:data-action-cost-ablation-swing-duration} that
under the same test parameters (10\% terrain difficulty and 0.15\,m/s
commanded velocity), the Cost-Ablated-GaitNet takes less redundant
steps\textemdash when the same foot is moved multiple times in succession.

\begin{figure}[H]
  \centering
  \includegraphics[width=1\textwidth]{images/data/swing-schedule-cost-ablation-comparison-d0.1-v0.15.png}
  \caption{Comparison of swing schedules for GaitNet (top)
    and Cost-Ablated-GaitNet (bottom) at 10\% terrain difficulty with a
  commanded velocity of 0.15\,m/s.}
  \label{fig:data-action-cost-ablation-swing-duration}
\end{figure}

These findings suggest that the footstep candidate cost is  hindering
GaitNet's ability to learn an optimal gait policy. By relying on this
cost term, GaitNet may be constrained in its exploration of the
action space, leading to suboptimal performance. Removing this term
allows for more freedom to explore and learn effective gait policies.
