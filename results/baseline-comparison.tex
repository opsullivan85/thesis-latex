\section{Baseline Comparison}
\label{sec:results-baseline-comparison}

For the purpose of evaluating GaitNet's performance, a baseline
method is established using a single leg motion planner. This planner
operates as is described in \cite{bratta_contactnet_2024}, where in
place of ContactNet, we directly use our footstep evaluation network
described in \autoref{sec:methodology-footstep-evaluation-network},
excluding the noise post-processing step
(\autoref{fig:diagram-costmap-processing-noise}), and only picking
one candidate in the selection step
(\autoref{fig:diagram-costmap-processing-topk}). The lowest cost
candidate is then used as the footstep target for that leg, with a
200ms swing duration.

In order to benchmark the two methods, a custom test environment is
created (\autoref{fig:figure-test-environment}). This environment has
the robots navigate straight forward across narrow strips of terrain
with characteristics matching the sub-terrains used in training
(\autoref{sec:methodology-simulation-environment}).  The space of
terrain difficulties and commanded forward velocities is
systematically explored with a grid search, evaluating the success
rate of each method over 150 trials of 20\,s each. An episode is
considered successful if the robot is able to complete the full 20\,s
without terminating according to the conditions described in
\autoref{sec:appendix-termination-functions}.

\begin{figure}[H]
  \centering
  \includegraphics[width=\textwidth]{images/figures/test-environment.png}
  \caption{Three robots simultaneously navigating a test environment
    (40\% terrain difficulty) used for baseline comparison. The terrain
    consists of narrow strips of a grid pattern with missing sections.
    Density of missing sections (difficulty) and commanded forward
  velocity are varied to evaluate performance.}
  \label{fig:figure-test-environment}
\end{figure}

\autoref{fig:data-experiments-survival-curr} and
\autoref{fig:data-experiments-baseline-method}
illustrate the performance of GaitNet and the baseline method,
respectively, across a range of terrain difficulties and commanded
velocities. The results indicate that GaitNet outperforms the
baseline in most scenarios, particularly on more challenging terrains
and at higher commanded speeds. This demonstrates the effectiveness
of GaitNet in generating robust gaits capable of adapting to varying conditions.

A closer examination of the graphs reveals that GaitNet maintains
strong performance when the commanded velocity is below 0.15\,m/s or
terrain difficulty is under 5\%. In contrast, ContactNet's
performance begins to degrade for commanded velocities above
0.05\,m/s and deteriorates further as terrain difficulty increases.
GaitNet's superior performance in these scenarios underscores the
benefits of its dynamic, acyclic gait generation capabilities.

\begin{figure}[H]
  \centering
  \begin{minipage}[t]{0.45\textwidth}
    \centering
    \includegraphics[width=\textwidth]{images/data/experiments/Gaitnet
    - survival-curr - 69.4 - individual.png}
    \caption{GaitNet Evaluation. Overall survival rate of 69.4\%.
      Mean success rate measured as the percentage of 150 trials
      which completed 20\,s without terminating, under the termination
      conditions described in
      \autoref{sec:appendix-termination-functions}.   Data point shapes
    denote different training instances.}
    \label{fig:data-experiments-survival-curr}
  \end{minipage}
  \hfill
  \begin{minipage}[t]{0.45\textwidth}
    \centering
    \includegraphics[width=\textwidth]{images/data/experiments/Contactnet
    - baseline-method - 25.6 - individual.png}
    \caption{Single Leg Motion Planner Evaluation. Overall survival
      rate of 25.6\%. Mean success rate measured as the percentage of 150 trials
      which completed 20\,s without terminating, under the termination
    conditions described in \autoref{sec:appendix-termination-functions}.}
    \label{fig:data-experiments-baseline-method}
  \end{minipage}
  \hfill
  % \caption{Evaluation of GaitNet vs. single leg motion planner   %
  %  across various terrain difficulties and commanded velocities.
  % %   Mean success rate measured as the percentage of 150 trials
  % %   which completed 20\,s without terminating, under the
  % termination   % conditions described in
  % \autoref{sec:appendix-termination-functions}.}   %
  % \label{fig:data-terrain-evaluation-comparison}
\end{figure}

% \begin{figure}[H] %   \centering %
% \includegraphics[width=\textwidth]{images/data/terrain-evaluation-comparison.png}
% %   \caption{Evaluation of GaitNet vs. single leg motion planner %
%    across various terrain difficulties and commanded velocities. %
%    Mean success rate measured as the percentage of 150 trials %
% which completed 20\,s without terminating, under the termination %
%  conditions described in
% \autoref{sec:appendix-termination-functions}.} %
% \label{fig:data-terrain-evaluation-comparison} % \end{figure}
