\section{Baseline Comparison}
\label{sec:results-baseline-comparison}

For the purpose of evaluating GaitNet's performance, a baseline
method is established using a single leg motion planner. This planner
operates as is described in \cite{bratta_contactnet_2024}, where in
place of ContactNet, we directly use our footstep evaluation network
described in \autoref{sec:methodology-footstep-evaluation-network},
excluding the noise post-processing step
(\autoref{fig:diagram-costmap-processing-noise}), and only picking
one candidate in the selection step
(\autoref{fig:diagram-costmap-processing-topk}). The lowest cost
candidate is then used as the footstep target for that leg, with a
200ms swing duration.

In order to benchmark the two methods, a custom test environment is
created (\autoref{fig:figure-test-environment}). This environment has
the robots navigate straight forward across narrow strips of terrain
with characteristics matching the sub-terrains used in training
(\autoref{sec:methodology-simulation-environment}). The space of
terrain difficulties and commanded forward velocities is
systematically explored with a grid search, evaluating the success
rate of each method over \expBenchmarkEpisodes{} trials of
\expBenchmarkDuration{} each. The terrain difficulty
is a value between 0\% and 100\% represents the percent of terrain
which can be stepped on. A checkerboard pattern (with alternating
holes and valid terrain) would have a terrain difficulty of 50\%,
though in these tests the terrain is randomly distributed, so a
checkerboard pattern is unlikely. The chosen range of terrain
difficulties \{0.0\%, 0.05\%, 0.1\%, \dots, 0.4\%\} and command
velocities \{0.05\,m/s, 0.1\,m/s, 0.15\,m/s, 0.2\,m/s\} were chosen
to represent a wide variety of situations while not straying too far
from what any method is capable of.

An episode is considered successful if the robot is able to complete
the full \expBenchmarkDuration{} without terminating according to the
conditions described in
\autoref{sec:appendix-termination-functions}. Briefly, these
conditions amount to terminating if the robot falls over or has its
foot slip into the holes in the ground.

\begin{figure}[H]
  \centering
  \includegraphics[width=\textwidth]{images/figures/test-environment.png}
  \caption{Three robots simultaneously navigating a test environment
    (40\% terrain difficulty) used for baseline comparison. The terrain
    consists of narrow strips of a grid pattern with missing sections.
    Density of missing sections (difficulty) and commanded forward
  velocity are varied to evaluate performance.}
  \label{fig:figure-test-environment}
\end{figure}

\autoref{fig:data-experiments-survival-curr} and
\autoref{fig:data-experiments-baseline-method}
illustrate the performance of GaitNet and the baseline method,
respectively, across a range of terrain difficulties and commanded velocities.

\begin{figure}[H]
  \centering
  \includegraphics[width=\smallplotsize{}]{images/data/experiments/Gaitnet
  - survival-curr - 69.4 - individual.png}
  \caption{GaitNet Evaluation. Overall survival rate of
    \resGaitNetSurvivalRate{}.
    Mean success rate measured as the percentage of
    \expBenchmarkEpisodes{} trials which completed
    \expBenchmarkDuration{} without terminating, under the
    termination conditions described in
    \autoref{sec:appendix-termination-functions}. Data point shapes
  denote different training instances.}
  \label{fig:data-experiments-survival-curr}
\end{figure}

\begin{figure}[H]
  \centering
  \includegraphics[width=\smallplotsize{}]{images/data/experiments/Contactnet
  - baseline-method - 25.6 - individual.png}
  \caption{Single Leg Motion Planner Evaluation. Overall survival
    rate of \resBaselineSurvivalRate{}. Mean success rate measured as
    the percentage of \expBenchmarkEpisodes{} trials which completed
    \expBenchmarkDuration{} without terminating, under the
    termination conditions described in
  \autoref{sec:appendix-termination-functions}.}
  \label{fig:data-experiments-baseline-method}
\end{figure}

\subsection{Discussion}

GaitNet's performance relative to the single leg motion planner
baseline reveals the clear advantage of dynamic gait generation. The
\resGaitNetSurvivalRate{} survival rate achieved by GaitNet, compared
to \resBaselineSurvivalRate{} for the baseline, demonstrates that the
system's ability to coordinate multi-leg motions and adapt timing
significantly improves robustness. This improvement is particularly
pronounced at higher commanded velocities and terrain difficulties,
where the baseline's single leg motion strategy fails.

A closer examination of the graphs reveals that GaitNet maintains
strong performance when the commanded velocity is below 0.15\,m/s or
terrain difficulty is under 5\%. In contrast, ContactNet's
performance begins to degrade for commanded velocities above
0.05\,m/s and deteriorates further as terrain difficulty increases.
The performance gap narrows only in the easiest conditions (at
0.05~m/s velocity), suggesting that dynamic gait planning becomes
increasingly valuable as task complexity increases. GaitNet's
superior performance in these scenarios underscores the benefits of
its dynamic, acyclic gait generation capabilities.
