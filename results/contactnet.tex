\section{Footstep Evaluation Network}
\label{sec:results-footstep-evaluation-network}

The results of the footstep evaluation network are highly promising,
with the model able to predict footstep candidate maps with strong
accuracy. As a brief recap, the cost maps visualize a set of 
footstep candidates $\mathbf f_c$, which show the heuristic
cost of moving legs to different positions from an initial state.
\autoref{fig:data-cn-typical-comparison} shows the model
output alongside the ground truth for a typical data sample. There
are some very minor differences between the ground truth and model
generated cost maps, but the model is capable of correctly identifying
the general low-cost regions (darker). Notice how the ground truth generally
has the lowest cost regions at $y=0$ and $x=\pm0.07$ depending on the
side of the leg, and the model generated cost map identifies the
lowest cost areas as being in generally the same places.

In the context of this work, the precise accuracy of the model is not
critical. Its primary role is to sample the continuous space of foot
positions to generate candidate actions for the GaitNet policy, so the
most important aspect is correctly identifying the general lowest cost
regions to move the foot to.

\begin{figure}[H]
  \centering
  \begin{subfigure}[T]{\smallplotsize{}}
    \includegraphics[width=\textwidth]{images/data/training/contactnet/typical-expected.png}
    \caption{Heuristically calculated cost map.}
  \end{subfigure}
  \hfill
  \begin{subfigure}[T]{\smallplotsize{}}
    \centering
    \includegraphics[width=\textwidth]{images/data/training/contactnet/typical-calculated.png}
    \caption{Cost map generated by the footstep evaluation network.}
  \end{subfigure}
  \hfill
  \caption{Comparison of a typical heuristically calculated cost map (a ground
    truth used in network training) to the reconstructed cost map
  generated by the footstep evaluation network.}
  \label{fig:data-cn-typical-comparison}
\end{figure}

\autoref{fig:data-cn-challenging-comparison} illustrates a
particularly challenging sample, in which the model still
successfully identifies the most suitable positions for each leg.
Notably, the rear left leg must be positioned far from its nominal
location to maintain stability in this scenario, and the model
correctly predicts this adjustment. Again, when comparing these
figures, the most important aspect is that the darkest region of
each sub-plot is in roughly the correct area. Looking at these
two plots we see just that, indicating the footstep evaluation
network will be suitable for generating footstep candidates.

\begin{figure}[H]
  \centering
  \begin{subfigure}[T]{\smallplotsize{}}
    \centering
    \includegraphics[width=\textwidth]{images/data/training/contactnet/challenging-expected.png}
    \caption{Heuristically calculated cost map.}
  \end{subfigure}
  \hfill
  \begin{subfigure}[T]{\smallplotsize{}}
    \centering
    \includegraphics[width=\textwidth]{images/data/training/contactnet/challenging-calculated.png}
    \caption{Cost map generated by the footstep evaluation network.}
  \end{subfigure}
  \hfill
  \caption{Comparison of a particularly challenging heuristically
    calculated cost map (a ground
    truth used in network training) to the reconstructed cost map
  generated by the footstep evaluation network.}
  \label{fig:data-cn-challenging-comparison}
\end{figure}
