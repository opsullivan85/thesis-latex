\section{Footstep Evaluation Network}
\label{sec:results-footstep-evaluation-network}

The results of the footstep evaluation network are highly promising,
with the model able to predict footstep candidate maps with strong
accuracy. \autoref{fig:data-cn-typical-comparison} shows the model
output alongside the ground truth for a typical data sample. \hl{I
  have a feeling that the text does not elaborate enough on the
  information presented by figures. Talk about the figure. Talk about
  specifics. Talk about individual subfigures and compare them
  together. This is where you need to have a detailed discussion. This
  figure is the basis for your future analysis so it is important to
  clarify everything about it and teach the reader how to read it and
  interpret it. I know you have presented similar figures in the
  previous chapter, but it is important to review here again and treat
these figures as they are stand-alone figures.}

\autoref{fig:data-cn-challenging-comparison} illustrates a
particularly challenging sample, in which the model still
successfully identifies the most suitable positions for each leg.
Notably, the rear left leg must be positioned far from its nominal
location to maintain stability in this scenario, and the model
correctly predicts this adjustment.

In the context of this work, the precise accuracy of the model is not
critical. Its primary role is to sample the continuous space of foot
positions to generate candidate actions for the GaitNet policy.

\begin{figure}[H]
  \centering
  \begin{subfigure}[T]{\smallplotsize{}}
    \includegraphics[width=\textwidth]{images/data/training/contactnet/typical-expected.png}
    \caption{Heuristically calculated cost map.}
  \end{subfigure}
  \hfill
  \begin{subfigure}[T]{\smallplotsize{}}
    \centering
    \includegraphics[width=\textwidth]{images/data/training/contactnet/typical-calculated.png}
    \caption{Cost map generated by the footstep evaluation network.}
  \end{subfigure}
  \hfill
  \caption{Comparison of a typical heuristically calculated cost map (a ground
    truth used in network training) to the reconstructed cost map
  generated by the footstep evaluation network.}
  \label{fig:data-cn-typical-comparison}
\end{figure}

\begin{figure}[H]
  \centering
  \begin{subfigure}[T]{\smallplotsize{}}
    \centering
    \includegraphics[width=\textwidth]{images/data/training/contactnet/challenging-expected.png}
    \caption{Heuristically calculated cost map.}
  \end{subfigure}
  \hfill
  \begin{subfigure}[T]{\smallplotsize{}}
    \centering
    \includegraphics[width=\textwidth]{images/data/training/contactnet/challenging-calculated.png}
    \caption{Cost map generated by the footstep evaluation network.}
  \end{subfigure}
  \hfill
  \caption{Comparison of a particularly challenging heuristically
    calculated cost map (a ground
    truth used in network training) to the reconstructed cost map
  generated by the footstep evaluation network.}
  \label{fig:data-cn-challenging-comparison}
\end{figure}
