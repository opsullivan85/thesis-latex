\section{Footstep Evaluation Network}

The results of the Footstep Evaluation Network are very promising,
with the model able to
predict footstep cost maps with high accuracy.
\autoref{fig:data-cn-typical-comparison}
shows the model output and ground truth for typical data sample.
\autoref{fig:data-cn-challenging-comparison}
shows a particularly challenging data sample, and the model is still
able to identify the best positions for each leg,
particularly the back left leg, which needs to be far from the
nominal position to maintain stability in that state. As it is used
in this work,
the specific accuracy of this model is not critical, as it is only used
to sample the continuous space of foot positions to generate candidate
actions for the GaitNet policy.

\begin{figure}[H]
  \centering
  \begin{minipage}[T]{0.45\textwidth}
    \centering
    \includegraphics[width=\textwidth]{images/data/training/typical-expected.png}
  \end{minipage}
  \hfill
  \begin{minipage}[T]{0.45\textwidth}
    \centering
    \includegraphics[width=\textwidth]{images/data/training/typical-calculated.png}
  \end{minipage}
  \hfill

  \caption{Typical data samples showing calculated (left) and
  expected (right) quadruped images.}
  \label{fig:data-cn-typical-comparison}
\end{figure}

\begin{figure}[H]
  \centering
  \begin{minipage}[T]{0.45\textwidth}
    \centering
    \includegraphics[width=\textwidth]{images/data/training/challenging-expected.png}
  \end{minipage}
  \hfill
  \begin{minipage}[T]{0.45\textwidth}
    \centering
    \includegraphics[width=\textwidth]{images/data/training/challenging-calculated.png}
  \end{minipage}
  \hfill

  \caption{Particularly challenging data samples showing calculated (left) and
  expected (right) quadruped images.}
  \label{fig:data-cn-challenging-comparison}
\end{figure}
