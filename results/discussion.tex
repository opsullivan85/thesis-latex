\section{Discussion}
\label{sec:results-discussion}

\begin{todo}
  summarize findings from ablation studies and baseline comparison
\end{todo}

The results presented in this chapter demonstrate that GaitNet
successfully generates dynamic, acyclic gaits for quadruped
locomotion in challenging environments. Through systematic evaluation
and ablation studies, several key insights emerge regarding the
system's capabilities and design choices.

The footstep evaluation network
(\autoref{sec:results-footstep-evaluation-network}) effectively
samples the continuous action space, providing diverse and feasible
footstep candidates to GaitNet. While not perfectly accurate in all
scenarios, the network's ability to identify suitable foothold
regions—even in challenging configurations—validates its role as a
candidate generator rather than a precise predictor.

GaitNet's performance relative to the single leg motion planner
baseline (\autoref{sec:results-baseline-comparison}) reveals the
clear advantage of dynamic gait generation. The
\resGaitNetSurvivalRate survival rate achieved by GaitNet, compared
to \resBaselineSurvivalRate for the baseline, demonstrates that the
system's ability to coordinate multi-leg motions and adapt timing
significantly improves robustness. This improvement is particularly
pronounced at higher commanded velocities and terrain difficulties,
where the baseline's fixed, sequential leg motion strategy fails. The
performance gap narrows only in the easiest conditions (below 5\%
terrain difficulty and 0.05~m/s velocity), suggesting that dynamic
gait planning becomes increasingly valuable as task complexity increases.

The swing duration ablation study
(\autoref{sec:results-swing-ablation-study}) yields a surprising
result: removing dynamic swing duration control has minimal impact on
performance, with survival rates of \resGaitNetSurvivalRate versus
\resDurationAblatedSurvivalRate. This finding suggests two
possibilities. First, the current training environment may not
sufficiently challenge the system to exploit variable swing
timing—the static terrain and relatively low speed requirements may
not create scenarios where dynamic timing provides substantial
benefits. Second, the reward function may not adequately incentivize
optimal swing duration selection, causing the network to default to
near-constant timing regardless of capability. The observation that
trained GaitNet naturally converges to a mean swing duration of
\resGaitNetMeanSwingDuration~s with only modest variation (standard deviation of
\resGaitNetSTDSwingDuration) supports this interpretation. Future
work in more dynamic environments or with revised reward structures
may reveal greater utility for this feature.

The action cost ablation study
(\autoref{sec:results-action-cost-ablation-study}) provides perhaps
the most significant insight. Counterintuitively, removing the
footstep candidate cost from GaitNet's input improves performance
from \resGaitNetSurvivalRate to \resCostAblatedSurvivalRate survival
rate. This result challenges the initial assumption that providing
the learned heuristic costs would accelerate training and improve
final performance. Several factors likely contribute to this outcome.
The correlation analysis (\autoref{fig:data-action-cost-correlation})
shows that while GaitNet initially relies heavily on the cost term
(correlation near 0.8), this dependence decreases during training,
suggesting the network learns to prioritize other state features. The
inclusion of the cost term may constrain exploration, biasing the
network toward locally optimal solutions that align with the footstep
evaluation network's heuristics but prevent discovery of globally
better strategies.

Moreover, Cost-Ablated-GaitNet exhibits a naturally slower cadence
(\autoref{fig:data-action-cost-ablation-steps-per-second}) and fewer
redundant footsteps
(\autoref{fig:data-action-cost-ablation-swing-duration}), indicating
more deliberate and efficient gait patterns. This behavior emerges
without explicit reward shaping, suggesting that the model better
captures the underlying dynamics when not constrained by pre-computed
costs. The slower initial learning observed in Cost-Ablated-GaitNet
represents a worthwhile trade-off for superior final performance, as
the network must learn terrain-aware footstep evaluation from scratch
rather than bootstrapping from the footstep evaluation network.

These findings collectively suggest that GaitNet's strength lies in
its ability to learn coordinated, multi-leg motion strategies through
reinforcement learning. The hierarchical architecture successfully
constrains the action space to valid movements while allowing
sufficient flexibility for dynamic gait generation. However, the
system appears to benefit more from direct exploration of
state-action relationships than from intermediate learned heuristics.
This insight has important implications for hybrid control
architectures: while learned perception modules can effectively
filter candidate actions based on geometric and kinematic
constraints, value estimation may be better left to end-to-end
learning that captures the full dynamics of the system.

The results also highlight opportunities for improvement. The
relatively modest absolute performance (\resCostAblatedSurvivalRate
best-case survival rate) indicates room for enhancement through
reward function refinement, network architecture modifications, or
extended training. The limited impact of dynamic swing duration
suggests that either the feature requires more sophisticated use
cases to demonstrate value, or the current implementation does not
adequately expose its potential benefits. Future work incorporating
prediction horizons or more complex terrain may better exploit this capability.

Overall, this work demonstrates that greedy, neural network-based
planners can generate effective dynamic gaits for quadruped robots.
The ablation studies provide valuable insights into which components
contribute most to performance, guiding future development of hybrid
locomotion controllers.
