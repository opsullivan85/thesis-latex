\section{GaitNet}
\label{sec:gaitnet}

\begin{todo}
  this section will show the results of training GaitNet.
  specifically, we need to show that the model is able to generate
  dynamic and acyclic gaits. We will not be comparing with the
  baseline   method in this section.
\end{todo}

The primary objective of GaitNet is to generate fully dynamic and
acyclic gaits for quadruped robots. An example swing sequence is
shown in \autoref{fig:data-swing-schedule}, illustrating a
non-repetitive, dynamic gait in which the robot performs motions with
up to two feet off the ground simultaneously. Swing durations are
non-uniform, with some legs remaining in the swing phase longer than
others. These results demonstrate that the system can synthesize a
wide variety of actions, which are analyzed further below.

Examining \autoref{fig:data-swing-schedule}, a short step is observed
for the front-left leg (green) at 3\,s, while the rear-right leg
(red) executes a longer step at 5\,s. Although the differences in
duration are subtle, they indicate that the system is capable of
dynamically varying step timing.

Another important observation from \autoref{fig:data-swing-schedule}
is the system's response to changing control inputs. Between
0-7.5\,s, the robot is commanded to follow a slow input, and the
system takes a cautious approach, moving one foot at a time. After
7.5\,s, the input speed increases, prompting a more dynamic gait.
During this faster phase, the robot occasionally executes two steps
simultaneously and does not follow a strict alternating pattern,
further highlighting the system's flexibility in generating acyclic gaits.

\begin{todo}
  Modify figure \autoref{fig:data-swing-schedule} to show   the
  terrain/robot at multiple time stamps.
\end{todo}

\begin{figure}[H]
  \centering
  \includegraphics[width=\textwidth]{images/data/swing-schedule.png}
  \caption{Example swing schedule for a single gait cycle. Each row
  represents a leg, with color indicating the leg is in swing phase.}
  \label{fig:data-swing-schedule}
\end{figure}

\begin{todo}
  Include figure showing the distribution of swing durations.
\end{todo}

\begin{todo}
  Include training figures from tensorboard detailing how the model
  improves over time.
\end{todo}
