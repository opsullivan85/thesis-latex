\section{GaitNet}
\label{sec:results-gaitnet}

The primary objective of GaitNet is to generate fully dynamic and
acyclic gaits for quadruped robots. An example swing sequence is
shown in \autoref{fig:data-swing-schedule}, illustrating a
acyclic, dynamic gait in which the robot performs motions with
up to two feet off the ground simultaneously. Swing durations are
non-uniform, with some legs remaining in the swing phase longer than
others. A video of GaitNet navigating challenging terrain can be seen
\href{https://youtu.be/vJ3QgSC8hrI}{here}.

Examining \autoref{fig:data-swing-schedule}, a short step is observed
for the front-left leg (green) at 3\,s, while the rear-right leg
(red) executes a longer step at 5\,s. Additionally, between
0--7.5\,s, the robot is commanded to follow a slow input, moving one
foot at a time. After 7.5\,s, the input speed increases, prompting a
more dynamic gait. During this faster phase, the robot occasionally
executes two steps simultaneously and does not follow a strict
alternating pattern.

\begin{figure}[H]
  \centering
  \includegraphics[width=\textwidth]{images/data/swing-schedule.png}
  \caption{Example swing schedule for a single gait cycle. Each row
  represents a leg, with color indicating the leg is in swing phase.}
  \label{fig:data-swing-schedule}
\end{figure}

To ensure the consistency of the training process, GaitNet was
trained three times. Due to computational limitations, the 700-step
training curriculum needed to be split into multiple runs. The runs
were typically 350 steps long. This has the side effect of resetting
the distribution of terrain levels, which can cause discontinuities
in certain training metrics. Resetting the simulations did not appear
to have an adverse affect on the learning process.

The mean episode reward during training is shown in
\autoref{fig:data-training-mean-reward}. Each training instance shows a
similar trend, with the reward increasing as the model learns to
navigate the terrain.

\begin{figure}[H]
  \centering
  \includegraphics[width=\smallplotsize{}]{images/data/training/gaitnet/mean_reward_survival-curr.png}
  \caption{Mean episode reward during GaitNet training. Each color
  represents a different training instance.}
  \label{fig:data-training-mean-reward}
\end{figure}

Another important metric is the mean steps per second, shown in
\autoref{fig:data-training-steps-per-second}. Initially,
with limited training, the model moves its legs very frequently,
leading to unstable motion. In initial experiments, it was found that
this issue does not resolve well without guidance from the reward function.
\autoref{sec:appendix-reward-function-analysis} provides more detail
on this issue. The mean steps per second starts high and decreases as
the model learns to take more stable steps.

\begin{figure}[H]
  \centering
  \includegraphics[width=\smallplotsize{}]{images/data/training/gaitnet/steps_per_second_survival-curr.png}
  \caption{Mean steps per second during GaitNet training. Each color
  represents a different training instance.}
  \label{fig:data-training-steps-per-second}
\end{figure}

During training, the GaitNet model learns to adjust swing durations
based on its input, as illustrated in
\autoref{fig:data-duration-mean} and \autoref{fig:data-duration-std}.
The mean swing duration settles to \resGaitNetMeanSwingDuration{}
with a standard deviation of \resGaitNetSTDSwingDuration{}.
\autoref{fig:data-swing-duration-histogram} shows exactly what has
been learned more clearly. For this histogram all swing durations
from a GaitNet test described in
\autoref{sec:results-baseline-comparison} were recorded. We can see a
large peak between 0.225\,s and 0.25\,s, with a significant amount of
swings down to 0.125\,s.

\begin{figure}[H]
  \centering
  \includegraphics[width=\smallplotsize{}]{images/data/training/gaitnet/duration_mean_survival-curr.png}
  \caption{Mean swing duration across all environments during
  GaitNet training. Each color represents a different training instance.}
  \label{fig:data-duration-mean}
\end{figure}

\begin{figure}[H]
  \centering
  \includegraphics[width=\smallplotsize{}]{images/data/training/gaitnet/duration_std_survival-curr.png}
  \caption{Swing duration standard deviation across all
    environments during GaitNet training. Each color represents a
  different training instance.}
  \label{fig:data-duration-std}
\end{figure}

\begin{figure}[H]
  \centering
  \includegraphics[width=\textwidth]{images/data/swing_duration_histogram.png}
  \caption{Histogram of swing durations selected by GaitNet.}
  \label{fig:data-swing-duration-histogram}
\end{figure}

\subsection{Discussion}

The results presented demonstrate that GaitNet successfully generates
dynamic, acyclic gaits for quadruped locomotion in challenging
environments. The system can synthesize a wide variety of actions,
and the results from the swing schedule analysis indicate that the
system is capable of dynamically varying step timing. The system's
response to changing control inputs\textemdash taking a cautious
approach at low speeds and executing simultaneous steps at high
speeds\textemdash highlights the system's flexibility in generating
acyclic gaits without pre-defined patterns.

Regarding the training metrics, there was no specific target for an
ideal steps per second, but the change during training shows that the
model is correctly learning when and when not to step. Similarly,
there was no specific target value for the swing duration metrics.
They are presented to verify that GaitNet is capable of learning to
utilize its swing duration output. The convergence of the mean swing
duration and standard deviation indicates that the model uses a wide
range of its allowable swing durations. The histogram further
clarifies this, showing that while the system favors durations
between 0.225\,s and 0.25\,s for normal motion, it utilizes durations
down to 0.125\,s for situational motions, such as moving across
tricky terrain or recovering from falls. These figures confirm that
the model learns to effectively utilize dynamic swing durations to
adapt to varying conditions.
