\section{GaitNet}
\label{sec:results-gaitnet}

The primary objective of GaitNet is to generate fully dynamic and
acyclic gaits for quadruped robots. An example swing sequence is
shown in \autoref{fig:data-swing-schedule}, illustrating a
non-repetitive, dynamic gait in which the robot performs motions with
up to two feet off the ground simultaneously. Swing durations are
non-uniform, with some legs remaining in the swing phase longer than
others. These results demonstrate that the system can synthesize a
wide variety of actions, which are analyzed further below. A video of
GaitNet navigating challenging terrain can be seen
\href{https://youtu.be/vJ3QgSC8hrI}{here}.

Examining \autoref{fig:data-swing-schedule}, a short step is observed
for the front-left leg (green) at 3\,s, while the rear-right leg
(red) executes a longer step at 5\,s. Although the differences in
duration are subtle, they indicate that the system is capable of
dynamically varying step timing.

Another important observation from \autoref{fig:data-swing-schedule}
is the system's response to changing control inputs. Between
0--7.5\,s, the robot is commanded to follow a slow input, and the
system takes a cautious approach, moving one foot at a time. After
7.5\,s, the input speed increases, prompting a more dynamic gait.
During this faster phase, the robot occasionally executes two steps
simultaneously and does not follow a strict alternating pattern,
further highlighting the system's flexibility in generating acyclic gaits.

\begin{figure}[H]
  \centering
  \includegraphics[width=\textwidth]{images/data/swing-schedule.png}
  \caption{Example swing schedule for a single gait cycle. Each row
  represents a leg, with color indicating the leg is in swing phase.}
  \label{fig:data-swing-schedule}
\end{figure}

To ensure the consistency of the training process, GaitNet was
trained three times. Due to computational limitations, the 700-step
training curriculum needed to be split into multiple runs. The runs
were typically 350 steps long. This has the side effect of resetting
the distribution of terrain levels, which can cause discontinuities
in certain training metrics. Resetting the simulations did not appear
to have an adverse affect on the learning process.

The mean episode reward during training is shown in
\autoref{fig:data-training-mean-reward}. Each training instance shows a
similar trend, with the reward increasing as the model learns to
navigate the terrain.

\begin{figure}[H]
  \centering
  \includegraphics[width=\smallplotsize{}]{images/data/training/gaitnet/mean_reward_survival-curr.png}
  \caption{Mean episode reward during GaitNet training. Each color
  represents a different training instance.}
  \label{fig:data-training-mean-reward}
\end{figure}

Another important metric is the mean steps per second. Initially,
with limited training, the model moves its legs very frequently,
leading to unstable motion. In initial experiments, it was found that
this issue does not resolve well without guidance from the reward function.
\autoref{sec:appendix-reward-function-analysis} provides more detail
on this issue, but the results are shown in
\autoref{fig:data-training-steps-per-second}. The mean steps per second
starts high and decreases as the model learns to take more stable
steps. There was no specific target for an ideal steps per second,
but the change during training shows that the model is correctly
learning when and when not to step.

\begin{figure}[H]
  \centering
  \includegraphics[width=\smallplotsize{}]{images/data/training/gaitnet/steps_per_second_survival-curr.png}
  \caption{Mean steps per second during GaitNet training. Each color
  represents a different training instance.}
  \label{fig:data-training-steps-per-second}
\end{figure}

During training, the GaitNet model learns to adjust swing durations
based on its input, as illustrated in
\autoref{fig:data-duration-mean} and \autoref{fig:data-duration-std}.
The mean swing duration settles to \resGaitNetMeanSwingDuration{}
with a standard deviation of \resGaitNetSTDSwingDuration{},
indicating that the model uses a wide range of its allowable swing
durations. These figures highlight how the model learns to
effectively utilize dynamic swing durations to adapt to varying
conditions. Again, there was no specific target value for either of
these metrics. They were chosen to verify that GaitNet is capable of
learning to utilize it's swing duration output.

\begin{figure}[H]
  \centering
  \includegraphics[width=\smallplotsize{}]{images/data/training/gaitnet/duration_mean_survival-curr.png}
  \caption{Mean swing duration across all environments during
  GaitNet training. Each color represents a different training instance.}
  \label{fig:data-duration-mean}
\end{figure}

\begin{figure}[H]
  \centering
  \includegraphics[width=\smallplotsize{}]{images/data/training/gaitnet/duration_std_survival-curr.png}
  \caption{Swing duration standard deviation across all
    environments during GaitNet training. Each color represents a
  different training instance.}
  \label{fig:data-duration-std}
\end{figure}

We can see from the figures above that GaitNet is learning to use its
swing duration output, but
\autoref{fig:data-swing-duration-histogram} shows exactly what has
been learned more clearly. For this histogram all swing durations
from a GaitNet test described in
\autoref{sec:results-baseline-comparison} were recorded. We can see a
large peak between 0.225\,s and 0.25\,s, this indicates that the
system uses swing durations in that range for normal motion. We also
see a significant amount of swings all the way down to 0.125\,s.
These seem to correspond to more situational motions, moving across
tricky terrain, or attempting to recovery from a fall.

\begin{figure}[H]
  \centering
  \includegraphics[width=\textwidth]{images/data/swing_duration_histogram.png}
  \caption{Histogram of swing durations selected by GaitNet.}
  \label{fig:data-swing-duration-histogram}
\end{figure}

\hl{It is not clear what parameters or specifications you are
  originally looking for and why. You present mean swing duration, mean
  reward, etc., but what are your metrics and how and why did you pick
  those metrics? What is the ideal scenario? Explain these in detail
early in this subsection (4.2)}
