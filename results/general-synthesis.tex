\section{General Synthesis of Results}

These findings collectively suggest that GaitNet's strength lies in
its ability to learn coordinated, multi-leg motion strategies through
reinforcement learning. The hierarchical architecture successfully
constrains the action space to valid movements while allowing
sufficient flexibility for dynamic gait generation. However, the
system appears to benefit more from direct exploration of
state-action relationships than from intermediate learned heuristics.
This insight has important implications for hybrid control
architectures: while learned perception modules can effectively
filter candidate actions based on geometric and kinematic
constraints, value estimation may be better left to end-to-end
learning that captures the full dynamics of the system.

The results also highlight opportunities for improvement. The
relatively modest absolute performance (\resCostAblatedSurvivalRate{}
best-case survival rate) indicates room for enhancement through
reward function refinement, network architecture modifications, or
improved low level control. Future work incorporating prediction
horizons or more complex terrain may better exploit these
capabilities. Overall, this work demonstrates that greedy, neural
network-based planners can generate effective dynamic gaits for
quadruped robots.
