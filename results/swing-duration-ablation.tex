\section{Swing Duration Ablation Study}
\label{sec:results-swing-ablation-study}

In this section we present an ablation study to assess the impact of
dynamic swing duration on GaitNet's performance. We compare two
models: the standard GaitNet formulation as described in
\autoref{sec:methodology-gaitnet}, and a Duration-Ablated-GaitNet trained
with a fixed swing duration.

\begin{todo}
  update the swing duration and standard deviations
\end{todo}

During training, the GaitNet model learns to adjust swing durations
based on its input, as illustrated with
\autoref{fig:data-duration-mean} and \autoref{fig:data-duration-std}.
The mean swing duration settles to 0.21\,s with a standard deviation
of 0.025\,s, indicating that the model uses a wide range of its
allowable swing durations. These figures highlight how the model
learns to effectively utilize dynamic swing durations to adapt to
varying conditions.

\begin{figure}[H]
  \centering
  \begin{minipage}[T]{0.45\textwidth}
    \centering
    \includegraphics[width=\textwidth]{example-image-a}
    \caption{Mean swing duration across all environments during
    GaitNet training.}
    \label{fig:data-duration-mean}
  \end{minipage}
  \hfill
  \begin{minipage}[T]{0.45\textwidth}
    \centering
    \includegraphics[width=\textwidth]{example-image-b}
    \caption{Swing duration standard deviation across all
    environments during GaitNet training.}
    \label{fig:data-duration-std}
  \end{minipage}
  \hfill
\end{figure}

Furthermore, a histogram of swing durations selected by GaitNet
during training, shown in \autoref{fig:data-duration-histogram},
reveals a broad distribution of values. This indicates that the model
is not biased towards a narrow range of swing durations, but rather
effectively leverages the flexibility provided by dynamic swing
durations to optimize its gait.

\begin{figure}[H]
  \centering
  \includegraphics[width=0.45\textwidth]{example-image-a}
  \caption{Histogram of swing durations selected by GaitNet
  during training, showing a wide distribution of values.}
  \label{fig:data-duration-histogram}
\end{figure}

Now, we train the Duration-Ablated-GaitNet model, setting the swing
duration to a constant 0.21\,s. The training process is identical to
that of the standard GaitNet model, ensuring a fair comparison. The
performance of both models is then evaluated using the same metrics outlined in
\autoref{sec:results-baseline-comparison}.

\textit{RESULTS IN PROGRESS}

% \begin{figure}[H] %   \centering %
% \includegraphics[width=\textwidth]{example-image-a} %
% \caption{Evaluation of GaitNet vs. Duration-Ablated-GaitNet %
% across various terrain difficulties and commanded velocities. %
% Mean success rate measured as the percentage of 50 episodes %
% which completed 20\,s without terminating, under the termination %
%  conditions described in
% \autoref{sec:appendix-termination-functions}.} %
% \label{fig:data-duration-ablation-comparison} % \end{figure}

% \autoref{fig:data-duration-ablation-comparison} presents the %
% performance comparison between the standard GaitNet and the %
% duration-ablated variant. The results indicate that the standard %
% GaitNet outperforms the duration-ablated model across a range of %
% terrain difficulties and commanded velocities. This demonstrates
% the % effectiveness of dynamic swing duration in enhancing
% GaitNet's % ability to adapt to varying conditions and generate robust gaits.
