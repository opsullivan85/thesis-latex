\section{Swing Duration Ablation Study}
\label{sec:results-swing-ablation-study}

In this section we present an ablation study to assess the impact of
dynamic swing duration on GaitNet's performance. We compare two
models: the standard GaitNet formulation as described in
\autoref{sec:methodology-gaitnet}, and a Duration-Ablated-GaitNet trained
with a fixed swing duration.

\begin{todo}
  tie back to the swing duration histogram
\end{todo}

Now, we train the Duration-Ablated-GaitNet model, setting the swing
duration to a constant 0.24\,s. The training process is identical to
that of the standard GaitNet model, ensuring a fair comparison. The
performance of both models is then evaluated using the same metrics outlined in
\autoref{sec:results-baseline-comparison}.

\begin{figure}[H]
  \centering
  \includegraphics[width=0.45\textwidth]{images/data/experiments/Gaitnet
  - swing-duration - 67.2 - individual.png}
  \caption{Evaluation of Duration-Ablated-GaitNet
    across various terrain difficulties and commanded velocities.
    Overall survival rate of 67.2\%.     Mean success rate measured
    as     the percentage of 50 episodes     which completed 20\,s
    without     terminating, under the termination   conditions described in
  \autoref{sec:appendix-termination-functions}.}
  \label{fig:data-duration-ablation-comparison}
\end{figure}

\autoref{fig:data-duration-ablation-comparison} presents the
performance of the Duration-Ablated-GaitNet. The results indicates
negligible performance changes over standard GaitNet formulation.
GaitNet was able to achieve an overall survival rate of 69.4\%
(\autoref{fig:data-experiments-survival-curr}), while the
Cost-Ablated-GaitNet achieved a survival rate of 67.2\%
(\autoref{fig:data-duration-ablation-comparison}). This suggests that
GaitNet may not be using the dynamic swing duration feature
effectively, or it may not be an important consideration in the
static environment being tested in.

This conclusion is further supported by the mean episode reward
during training, as shown in
\autoref{fig:data-duration-ablation-reward}. The steady state
behavior of the reward curves for both models indicates that the
Duration-Ablated-GaitNet is able to learn a policy that performs
comparably to the standard GaitNet, despite the lack of dynamic swing duration.

\begin{figure}[H]
  \centering
  \includegraphics[width=0.45\textwidth]{images/data/training/gaitnet/mean_reward_survival-curr_vs_swing-duration.png}
  \caption{Mean episode reward during training for GaitNet and
  Duration-Ablated-GaitNet.}
  \label{fig:data-duration-ablation-reward}
\end{figure}

These findings suggest that dynamic swing duration may not be a
critical factor for GaitNet's performance in the tested environments.
However, it is important to note that this conclusion may not hold in
more complex or dynamic environments, where the ability to adjust
swing duration could provide a significant advantage. Future work
should explore the impact of dynamic swing duration in a wider range
of environments and conditions
