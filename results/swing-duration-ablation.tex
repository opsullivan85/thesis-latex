\section{Swing Duration Ablation Study}
\label{sec:results-swing-ablation-study}

In this section we present an ablation study to assess the impact of
dynamic swing duration on GaitNet's performance. We compare two
models: the standard GaitNet formulation as described in
\autoref{sec:methodology-gaitnet}, and a Duration-Ablated-GaitNet trained
with a fixed swing duration. In the GaitNet formulation described
previously, the model explicitly outputs a swing duration to use for
every action. The corresponding swing duration is then used for the
chosen action. For this ablation, we ignore the swing duration output
of GaitNet and directly use a fixed value,
\expDurationAblatedSwingDuration{}. This value was chosen because it
is roughly the mean step duration used by GaitNet. This will provide
information about how important the variable swing duration is for
this use case.

\begin{todo}
  tie back to the swing duration histogram
\end{todo}

Now, we train the Duration-Ablated-GaitNet model, setting the swing
duration to a constant \expDurationAblatedSwingDuration{}. The
training process is identical to that of the standard GaitNet model,
ensuring a fair comparison. The performance of both models is then
evaluated using the same metrics outlined in
\autoref{sec:results-baseline-comparison}.

\begin{figure}[H]
  \centering
  \includegraphics[width=\smallplotsize{}]{images/data/experiments/Gaitnet
  - swing-duration - 67.2 - individual.png}
  \caption{Evaluation of Duration-Ablated-GaitNet
    across various terrain difficulties and commanded velocities.
    Overall survival rate of \resDurationAblatedSurvivalRate{}. Mean
    success rate measured as the percentage of
    \expBenchmarkEpisodes{} episodes which completed
    \expBenchmarkDuration{} without terminating, under the
    termination conditions described in
  \autoref{sec:appendix-termination-functions}.}
  \label{fig:data-duration-ablation-comparison}
\end{figure}

\autoref{fig:data-duration-ablation-comparison} presents the
performance of the Duration-Ablated-GaitNet. The results indicates
negligible performance changes over standard GaitNet formulation.
GaitNet was able to achieve an overall survival rate of
\resGaitNetSurvivalRate{}
(\autoref{fig:data-experiments-survival-curr}), while the
Cost-Ablated-GaitNet achieved a survival rate of
\resDurationAblatedSurvivalRate{}
(\autoref{fig:data-duration-ablation-comparison}). The mean episode
reward during training, as shown in
\autoref{fig:data-duration-ablation-reward}, shows a similar trend.

\begin{figure}[H]
  \centering
  \includegraphics[width=\smallplotsize{}]{images/data/training/gaitnet/mean_reward_survival-curr_vs_swing-duration.png}
  \caption{Mean episode reward during training for GaitNet and
  Duration-Ablated-GaitNet.}
  \label{fig:data-duration-ablation-reward}
\end{figure}

\subsection{Discussion}

The swing duration ablation study yields a surprising result:
removing dynamic swing duration control has minimal impact on
performance. The steady state behavior of the reward curves for both
models indicates that the Duration-Ablated-GaitNet is able to learn a
policy that performs comparably to the standard GaitNet, despite the
lack of dynamic swing duration. This finding suggests two
possibilities. First, the current training environment may not
sufficiently challenge the system to exploit variable swing
timing\textemdash the static terrain and relatively low speed
requirements may not create scenarios where dynamic timing provides
substantial benefits. Second, the reward function may not adequately
incentivize optimal swing duration selection, causing the network to
default to near-constant timing regardless of capability.

The observation that trained GaitNet naturally converges to a mean
swing duration of \resGaitNetMeanSwingDuration{} with only modest
variation (standard deviation of
\resGaitNetSTDSwingDuration{}) supports this interpretation. These findings
suggest that dynamic swing duration may not be a critical factor for
GaitNet's performance in the tested environments. However, it is
important to note that this conclusion may not hold in more complex
or dynamic environments, where the ability to adjust swing duration
could provide a significant advantage. Future work in more dynamic
environments or with revised reward structures may reveal greater
utility for this feature.
